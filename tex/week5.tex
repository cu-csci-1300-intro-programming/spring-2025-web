\chapter*{Week 5: Objects}
\addcontentsline{toc}{chapter}{Week 5: Objects}
\setcounter{chapter}{6}
\setcounter{section}{0}

\begin{abstract}
This week will cover:
\begin{enumerate}
    \item Input from files
    \item Output to files
    \item Learn what an object is
    \item Be able to explain what ``Object Oriented" means
    \item Learn the difference between ``Public" and ``Private"
    \item Be able to explain:
    \begin{itemize}
        \item Member variables
        \item Member functions
        \item Constructors
    \end{itemize}
\end{enumerate}
    
\end{abstract}

\section{Background}
\subsection{File Input/Output}
So far in class, we've been using the \mintinline{c++}{iostream} standard library. This library has provided us with methods like \mintinline{c++}{cin} and \mintinline{c++}{cout}. \mintinline{c++}{cin} is the method that reads from standard input (i.e. in the terminal via the keyboard) and \mintinline{c++}{cout} is for writing to standard output.

In this background section we'll cover file input, which will allow you to read information from a file. To do this, we'll need to include C++'s fstream library, which is short for "file stream".

\subsubsection{Reading Lines From A File}

\textbf{Step 1. Make a stream object.}

Create an object (a variable) of file stream type. If you want to open a file for reading only, then the ifstream object should be used (short for “input file stream”).

\begin{minted}{c++}
// create an input file stream object
ifstream file_input;
\end{minted}

\textbf{Step 2. Open a file.}

Once you have a file stream object, you need to open the file. To do this, use the ifstream object's \mintinline{c++}{open()} method (function), which takes only one parameter: the file name as a string (surrounded by " " if the file name is given directly).

\begin{minted}{c++}
// open myTextFile.txt with the file stream object
file_input.open("myTextFile.txt");
\end{minted}
  
\textbf{Step 3. Checking for open files.}

It is always good practice to check if the file has been opened properly and take an appropriate action if not. To check if a file was successfully opened, you may use the \mintinline{c++}{fail()} or \mintinline{c++}{is_open()} methods.

\mintinline{c++}{fail()}: This method will return a boolean value true if the file failed to open and false otherwise.

\begin{minted}[breaklines=true]{c++}
if (file_input.fail()) // true when file fails to open
{
    cout << "Could not open file." << endl; 
    return -1; // return to terminate the program; -1 to indicate that the program didn't function as expected
}
// do things with the file
\end{minted}

\mintinline{c++}{is_open()}: This method will return a boolean value true if the file has successfully opened and false otherwise.

\begin{minted}[breaklines=true]{c++}
if (file_input.is_open()) // true when file opens sucessfully
{
    // do things with the file
}
else
{
    cout << "Could not open file." << endl;        
}
\end{minted}

\textbf{Step 4. Read lines from the file.}

To read a line from the file, you can use \mintinline{c++}{getline(file_input, line)} which returns true as long as an additional line has been successfully assigned to the variable line. Once no more lines can be read in, getline returns false. So we can set up a while loop where the condition is the call to getline. 

\mintinline{c++}{.eof()}: This method will return a boolean value true if all the data in the file was processed and false otherwise.

\begin{minted}[breaklines=true]{c++}
string line = "";
int line_idx = 0;
// read each line from the file
while (!file_input.eof()) // continue looping as long as there is data to be processed in the file
{
    // get the next line from the file and store in 'line' variable
    getline(fin, line);

    // print each line read from the file
    cout << line_idx << ": " << line << endl;

    // increment index(count of lines in the file)
    line_idx++;   
}
\end{minted}

\textbf{Step 5. Closing a file.}

When you are finished processing your files, it is recommended to close all the opened files before the program is terminated. You can do this by using the \mintinline{c++}{.close()} function on your file stream object.

\begin{minted}{c++}
// closing the file
file_input.close();
\end{minted}


\textbf{Step 6. Putting it all together.}
If we put all the previous steps together, this is the final piece of code we get.

\begin{minted}[breaklines=true]{c++}
// create an input file stream object
ifstream file_input;

// open myTextFile.txt with the file stream object
file_input.open("myTextFile.txt"); 

// check if file opened successfully
if (file_input.fail())
{
    cout << "Could not open file." << endl;
    return -1;
}
else
{
    // do things with the file
    string line = "";
    int line_idx = 0;

    // read each line from the file
    while (!file_input.eof())
    {
        // gets line of text from file_input, stores it in line
        getline(file_input, line);

        // print each line read from the file
        cout << line_idx << ": " << line << endl;

        // increment index (count of lines in the file)
        line_idx++;   
    }
}

// closing the file
myTextFile.close();
\end{minted}

\subsection{Classes}
When writing complex programs, sometimes the built-in data types (such as int, char, string) don’t offer developers enough functionality or flexibility to solve their problems. A solution is for the developer (you - yes, you!) to create your own custom data types to use, called classes. Classes are user-defined data types, which hold their own data members and member functions, which can be accessed and used by creating an instance of that class. A class is like a blueprint for an object, customized for whatever particular problem the programmer is working on.

String, for example, is a class in C++ which holds data (the characters comprising the string) and supports useful member functions like substr which operate on this data.

Below is an example of the basic definition of a class in C++.

\begin{minted}{c++}
class ClassName{
    public:
        //The public interface
        //Member functions

    private:
        //The private data members
}; //must end with a semicolon
\end{minted}

\subsection{Anatomy of a Class}
\subsubsection{Class Name}
A class is defined in C++ using the keyword \mintinline{c++}{class} followed by the name of the class. The body of the class is defined inside the curly brackets and terminated by a semicolon at the end. This class name will be how you reference any variables or objects you create of that type. For example: 

\mintinline{c++}{ClassName objectName;}

The above line would create a new object (variable) with name \mintinline{c++}{objectName} and of type \mintinline{c++}{ClassName}, and this object would have all the properties that you have defined within the class \mintinline{c++}{ClassName}.

\subsubsection{Access Specifiers}
Access specifiers in a class are used to set the accessibility of the class members. That is, they restrict access by outside functions to the class members.

Consider the following analogy:

Imagine an intelligence agency like the CIA, that has 10 senior members. Such an organization holds various sorts of information, and needs some way of determining who has access to any given piece of information. Some information concerning classified or dangerous operations may only be accessible to the 10 senior members of the agency, and not directly accessible by any other person. This data would be \textbf{private}.

In a class, like in our intelligence agency, these private data members are only accessible by a class's member functions and not directly accessible by any object or function outside the class.

Some other information may be available to anyone who wants to know about it. This is \textbf{public} data. Even people outside the CIA can know about it, and the agency might release this information through press releases or other means. In terms of our class, this public data can be accessed by any member function of the class, as well as outside functions and objects, even other classes. The public members of a class can be accessed from anywhere in the program using the direct member access operator (.) with the object of that class.

\subsubsection{Data Members and Member Functions}
Data members are the data variables and member functions are the functions used to manipulate these variables; together, data members and member functions define the properties and behavior of the objects in a Class.

The data members declared in any class definition can be fundamental data types (like \mintinline{c++}{int}, \mintinline{c++}{char}, \mintinline{c++}{float}, etc.), \mintinline{c++}{arrays}, or even other classes.

For example, for string objects, the data member is a \mintinline{c++}{char array[]} that holds all of the individual letters of your string. Some of a string’s member functions that we have used already are functions like \mintinline{c++}{.substr()} and \mintinline{c++}{.length()}.

\subsubsection{Accessing Data Members}
Keeping with our intelligence agency example, the code below defines a class that holds information for any general agency. In the main function, we then create a new \mintinline{c++}{IntelligenceAgency} object called \mintinline{c++}{CIA}, and we set its \mintinline{c++}{organizationName} to “CIA” by using the access operator (.) and the corresponding data member’s name. However, we cannot access the \mintinline{c++}{classifiedIntelligence} data member in the same way. Not everyone has access to that private data.

Instead, we need to use a member function of the \mintinline{c++}{IntelligenceAgency} class, \mintinline{c++}{getClassifiedIntelligence()}, in order to see that information. This allows us to control the release of private information by our \mintinline{c++}{IntelligenceAgency}.

Additionally, the main function includes four different ways of creating objects with descriptions in the comments next to it.

\begin{minted}[breaklines=true]{c++}
class IntelligenceAgency
{
    public:
        IntelligenceAgency();         // Default constructor
        IntelligenceAgency(string classified_intelligence_input); // Parameterized constructor  
        string organizationName;
        string getClassifiedIntelligence();
        void setClassifiedIntelligence(string classifiedIntelligenceInput);

    private:
        string classifiedIntelligence;
};

int main()
{
    IntelligenceAgency CIA;
    CIA.organization_name = "CIA";
    cout << CIA.classifiedIntelligence; // gives an error
    cout << CIA.getClassifiedIntelligence();
    
    // four types of constructor calls
    IntelligenceAgency abc; // creating an IntelligenceAgency object with the default constructor 
    IntelligenceAgency xyz = IntelligenceAgency(); // creating an IntelligenceAgency 
    //object with the default constructor 
    IntelligenceAgency pqr("PQR"); // creating an IntelligenceAgency 
    //object with a paramaterized constructor
    IntelligenceAgency rst = IntelligenceAgency("RST"); // creating an IntelligenceAgency 
    //object with a paramaterized constructor
}
\end{minted}

\subsection{Defining Member Functions}
You may have noticed that we declared various member functions in our class definition, but we did not specify how they will work when called. The body of the function still needs to be written. The solution is to define a function, such as \mintinline{c++}{getClassifiedIntelligence()}, corresponding to our class’s functions. But how does our program know that these functions are the same as the ones we declared in our class? You as the developer need to explicitly tell the computer that these functions you are defining are the same ones you declared earlier.

\begin{minted}[breaklines=true]{c++}
string IntelligenceAgency::getClassifiedIntelligence()
{
    return classifiedIntelligence;
}
void IntelligenceAgency::setClassifiedIntelligence(string classifiedIntelligenceInput)
{
    classifiedIntelligence = classifiedIntelligenceInput;
}
\end{minted}

We start the definition as we do any function, with the return type. Then, we have to add a specifier \mintinline{c++}{IntelligenceAgency::} that lets our program know that this function and the one we declared inside the class are the same. We can see that this function returns the class’ \mintinline{c++}{classifiedIntelligence} to the user.

Functions like \mintinline{c++}{getClassifiedIntelligence()} are called accessor/getter functions. This is because they retrieve or `get’ the private data members of a class object and return it to the program so that these values may be used elsewhere.

The second function, \mintinline{c++}{setClassifiedIntelligence(string classifiedIntelligenceInput)}, is called a mutator/setter function. These allow functions from other parts of our program to modify or ‘set’ the private data members of our class objects. Getters and setters are the functions that our program uses to interact with the private data members of our class objects.

\section{Warmup}

\begin{problem}
Fill in the blank(s). The program needs to print all the contents of the file input.txt present in the current folder.
\begin{minted}[breaklines=true]{c++}
    #include<iostream>
    #include<iostream>
    #include<fstream>
    #include<string>
    using namespace std;
    
    int main() 
    {
        string fileName = "input.txt";
        // Fill the line below to make an input file stream from the file name given above.
        _____________________________________
        string text;  
        while(getline(infile, text))
        {      
            cout << text << endl;
        }
        return 0;
    }

\end{minted}

\end{problem}

\begin{problem}
Fill in the blank(s). We want to make a copy of the input file by writing each line of the input file to the output file.
\begin{minted}[breaklines=true]{c++}
    #include<iostream>
    #include<fstream>
    #include<string>
    using namespace std;
    
    int main() 
    {
        // Fill the line below to make an input file stream from the file input.txt.
        _______________________________
        // Fill the line below to make an output file stream to the file output.txt.
        _______________________________
        string text;  
        while(getline(infile, text))
        {
            // Fill the line below to write the text to the output file stream.
            _____________________________________
        }
        return 0;
    }
    \end{minted}
\end{problem}

\begin{problem}
    Identify the error(s).
    \begin{minted}{c++}
    class Rectangle
    {
        private:
            int height;
            int width;
        public:
            rectangle(int h=0, int w)
            {
                height = h;
                width = w;
            }
            int getArea()
            {
                return height * width;
            }
    };
    \end{minted}
\end{problem}

\begin{problem}
    Identify the error(s).
    \begin{minted}{c++}
    #include <iostream>
    #include <string>
    using namespace std;
    // This object can add two integers
    class Adder
    {
        private:
            int add(int first, int second)
            {
                return first + second;
            }
    }
    int main()
    {
        // ask user for 2 integers
        int first, second;
        cout << "Enter integers to add:" << endl;
        cin >> first >> second;
        // create the adder object
        Adder Adder;
        // add the integers
        cout << "The sum is " << adder.add(first, second) << endl;
        return 0;
    }
    \end{minted}
\end{problem}

\begin{problem}
    Identify the error(s).
    \begin{minted}{c++}
    class Square {
        private:
            int side;
        public:
            void Square(double x) {
                side = x;
            }
            int area() {
                return x * x;
            }
    };
    \end{minted}
\end{problem}

\section{Recitation}

\subsection{File Concatenation}
Motivating Example:

The cat command (short for ``concatenate") is one of the most frequently used commands in Linux and Mac operating systems. The cat command allows us to print concatenated contents of input files with the below command. It also allows us to combine copies of files, view contents of one or more files, and create new files.
\begin{minted}{c++}
cat file1.txt file2.txt
\end{minted}
Your task:

Write a program that concatenates the contents of two files and stores it in an output file. The program should ask the user to enter input and output filenames.

If one of the input files cannot be opened, then print ``Could not open file/files." If you can read files and concatenate them successfully, then print ``Files concatenated successfully".

Here are two sample input files:

\begin{minted}{c++}
input1.txt
Lorem ipsum dolor sit amet, consectetur adipiscing elit.
\end{minted}



\begin{minted}{c++}
input2.txt
Morbi nec felis pretium, elementum sem eu, tempor massa.
\end{minted}

\begin{sample}

Enter the input filenames that you want to concatenate:

\textcolor{red}{input1.txt}

\textcolor{red}{input2.txt}

Enter the output filename:

\textcolor{red}{output.txt}

Files concatenated successfully

\textcolor{red}{output.txt should have the following content:}

\textcolor{red}{Lorem ipsum dolor sit amet, consectetur adipiscing elit. Morbi nec felis pretium, elementum sem eu, tempor massa.}

\end{sample}


\begin{multipart}
You should create at least three sample input files to use for your own testing purposes, you may populate them with whatever you see fit. Write their names and their text content below:  
\end{multipart}

\vspace{3cm}

\begin{multipart}
Develop your solution in C++. Revise your solution, save, compile and run it again. Are you getting the expected result and output? Keep revising until you do.

Make sure you test for the values mentioned in the example output as well as your own test files.
\end{multipart}

\newpage


\subsection{The Point 2D Class}
Let’s design a class to represent a point in the 2D plane. The Point2D class will have the following attributes and methods:


Data Members (private)

\begin{table}[H]
    \centering
    \begin{tabular}{|p{1.5in}|p{1.5in}|p{1.5in}|} 
        \hline
        Name & Data Type & Description  \\ 
        \hline
        x & double & The x coordinate of the 2D point \\ 
        \hline
        y & double & The y coordinate of the 2D point \\ 
        \hline
        label & string & A label given to the point \\ 
        \hline
    \end{tabular}
\end{table}

Member Functions (public)

\begin{table}[H]
    \centering
    \begin{tabular}{|p{3in}|p{3in}|}
        \hline
        \textbf{Name} & \textbf{Description} \\ 
        \hline
        Default Constructor & Set x and y to 0.0 and label to empty string. \\ 
        \hline
        Parameterized Constructor & Set the values of x, y, label as provided to the constructor. label should have a default value of an empty string. \\ 
        \hline
        double getX() & Returns the value of x data member as a double \\ 
        \hline
        double getY() & Returns the value of y data member as a double \\ 
        \hline
        string getLabel() & Returns the value of label data member as a string \\ 
        \hline
        void setX(double) & Updates the value of x data member to function parameter \\ 
        \hline
        void setY(double) & Updates the value of y data member to function parameter \\ 
        \hline
        void setLabel(string) & Updates the value of label data member to function parameter \\ 
        \hline
        void print() & Formats and prints the contents of the Point2D object in the following format: label: (x,y). If the label string is empty, then simply print the coordinates in the following format: (x,y). For example, an object with x = 2, y = 3.5 and label = "A" will print the following string: A: (2,3.5) \\ 
        \hline
        double distance(Point2D) & Computes the Euclidean distance between the current object and the Point2D object passed as an argument, then returns the result as a double. The Euclidean distance between two 2D coordinates (x1, y1) and (x2, y2) is computed as follows: $\sqrt{(x1 - x2)^2 + (y1 - y2)^2}$. For an extra challenge, set the default parameter value to be the point at the origin, (0,0). \\ 
        \hline
    \end{tabular}

\end{table}


\textbf{Example main function:}
\begin{minted}{c++}
    Point2D p; //calls the default constructor
    p.print();
    Point2D q(3, 4, "Q");
    q.print();
    Point2D r(-10, 4); //uses the parameterized constructor
    r.print();
    r.setLabel("R"); //can also use setters to update data members
    r.print();
    cout << "q.distance(): " << q.distance() << endl;
    cout << "r.distance(q): " << r.distance(q) << endl;
    Expected Output:
    (0.0,0.0)
    (3.0,4.0): Q
    (-10.0,4.0)
    (-10.0,4.0): R
    q.distance(): 5
    r.distance(q): 13
\end{minted}


\begin{multipart}
You will need to test your code by providing some expected inputs and outputs. Come up with two sample runs where you create Point2D objects and test their functions.

\end{multipart}

\vspace{1.5cm}

\begin{multipart}
Develop your solution in C++. Revise your solution, save, compile and run it again. Are you getting the expected result and output? Keep revising until you do.

Make sure you test for the values mentioned in the example output, and also test your code on sample test cases inclusive of boundary conditions.
\end{multipart}


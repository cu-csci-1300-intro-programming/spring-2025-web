\chapter*{CSCI 1300 Syllabus}
\addcontentsline{toc}{chapter}{CSCI 1300 Syllabus}

\section{About The Course}
\subsection{Course Description}

Teaches techniques for writing computer programs in higher level programming languages to solve problems of interest in a range of application domains. Appropriate for students with little to no experience in computing or programming.

\subsection{Course Objectives}
This class covers the basics of computer programming using C++. The focus will be on understanding the basic components of computer languages. Upon completion of the course, the student can:
\begin{itemize}
    \item  Design and construct algorithms for problem solving by applying processes of abstraction and program decomposition.
    \item Implement fundamental programming constructs, such as variables, expressions, conditionals, and iterative control structures, in a higher-level language.
    \item Evaluate and implement simple I/O, such as user input and file I/O, including the necessity for external input to a program and the role of external data storage.
    \item Design and explain the role of functions in program construction, including an understanding of parameter passing and return values.
    \item Describe the properties of data types, including the primitive types of numbers, characters, and booleans, as well as more complex data types, such as arrays, records, and strings.
    \item Use an Integrated Development Environment to produce code that is free of syntactical, logical, and run-time errors. Understand the process of debugging as part of software development.
    \item Design and create code using object-oriented design methodology, including an understanding of objects and classes, information encapsulation, and efficient class design.
    \item Use third-party code to accomplish a programming objective. This includes learning to read code written by another individual and modifying the code, or using third-party libraries.
    \item Develop an understanding that software development is a dynamic, social process, and that learning how to seek out information is a necessary skill for success.
\end{itemize}

\subsection{Course Outline}

This is a 4 credit hour course. As such, you can be expected to spend 4 hours per week "in class" (3 in full class lecture, 1 in recitation) and 12 hours per week on additional course-related activities. These activities will include reading sections from the course text, watching pre-recorded topic videos, working on various assignments, engaging with course staff in grading interviews, and other activities.

Brief list of topics to be covered:
\begin{itemize}
    \item Computer architecture and environment
    \item Variables and data types
    \item Variables and operators
    \item Strings: indexing, iteration, comparing
    \item Control structures: if/else statements
    \item Control structures: switch statements
    \item Control structures: while loops
    \item Control structures: for loops, iteration
    \item Functions
    \item Arrays
    \item I/O Streams
    \item File I/O
    \item Classes and objects
    \item Classes, functions, methods
    \item Inheritance
    \item Vectors
    \item Recursion
    \item References
\end{itemize}

\section{Assignments and Grading}
\subsection{Assignments}
Your grade is based on your performance in course activities including recitations, homeworks, quizzes, projects, and exams. 

Your grade in the course will be based on the following components:

\begin{itemize}
    \item \textbf{Homeworks (25\%)}. There are 8 homework assignments that will asses your knowledge about the weekly concepts taught in lecture. 
    \item \textbf{Projects (15\%)}. You will complete two projects during the semester. Project 1 will be worth 5\% of your grade, and Project 2 will be worth 10\% of your grade. These assignments will build upon and integrate the concepts covered in previous coding assignments and give you opportunities to practice your skills on real-world problems. 
    \item \textbf{Weekly recitation activities (10\%, drop lowest)}. You will complete pre-quizzes before recitation and in-class worksheets during the recitation section. These are graded assignments and may often require working with a partner. Attendance in recitations is required.
    \item \textbf{Midterms and final (40\%)}. There will be three midterms and a final exam that will evaluate your overall knowledge of the course material. Dates for these, and topics covered, are listed in the table below. Midterms will be held in-person during lecture, and the final exam will be held in-person during the scheduled final exam time. You must be present in-person for all exams, they will not be offered remotely.
    \item \textbf{Class participation (10\%, drop 3 lowest)}. Most activities will take place during lecture. Other activities will be published via Canvas.  
\end{itemize}
    

\subsection{Extra Credit}

There will be a number of other opportunities to receive extra credit in the form of additional points on an assignment or within a category. Take advantage of each and every opportunity. 

\subsection{Submission Policies}

All assignments are due by the date and time specified in the assignment. Homework can be submitted up to two days late for 80\% of full credit. After two days, no late work is accepted without prior permission of the instructor.

\subsubsection{Recitations}

Recitation pre-quizzes will include questions to get you adjusted to the content taught in lecture the previous week. Pre-quizzes will be due at \textbf{9:00 am on Mondays}. Recitations will then address these questions and provide the correct answers, and then in recitation you will complete a paper worksheet provided by your TA. The worksheets will be due at \textbf{11:59 pm on Wednesdays}.

\subsubsection{Homeworks}

There is a homework or project due every week. Homework will be due at \textbf{5:00 pm on Fridays}. Coding assignments will often be given immediate feedback using an auto-grade feature, but may also involve a grading interview process outlined below. Code files should have your name, date, and assignment number included as comments at the top of the file.

There will be a quiz associated with every homework and project. These quizzes will include both theory and code questions. Quiz will be due at 5:00 pm on Sundays.

\subsubsection{Homework Late Policy:} All homework assignments, except Homework 0, may be turned in up to 2 days (48 hours) late with a 20\% penalty. Late extension homework coderunner assignments will be available at 5:01 pm on Fridays and may be submitted as late as Sunday at 5:00 pm. If a student would have received a 95\% had they turned in their homework on time, a late submission will earn them a 75\% instead. 

\subsubsection{Interview Grading:}

Following a project submission, each student will sign up for a grading interview about their project. If you are not interviewed for all projects, then you get no credit for those activities. The grading interview will consist of questions about the work you will have submitted the previous week (or two, for projects). These questions are designed to test your understanding of the solution code submitted. They will also serve as an opportunity for you check in about other topics related to the course. It is your responsibility to look on Canvas and sign-up for an interview slot. If you wait, and the interview slots dry-up, then you missed your interview. Sign-up early, do not wait.

If you have to reschedule a grading interview, you must email the TA at least one day in advance. Documented emergency situations will be evaluated on a case-by-case basis.

You are responsible for scheduling the grading interview during the designated grading period, which is usually 1 week. Once the period expires you will not be able to have the grading interview and you will lose all the points for the assignment. You are responsible for finding the room and arriving on time. There is a 1- minute “grace period” for being late, after that it is 10\% off for each minute the student is late, at 6 minutes late you get a zero.

Advice: Get to the appointment 5 to 10 minutes early and use the extra time to prepare. 

\subsubsection{Exams}

As discussed in the Assignments section above, there are three midterms and a final in this class that account for 40\% of your final grade. \textbf{You must average at least 67\% on your in order to receive better than a D+ in the class, regardless of your scores on the other aspects of the class}. Each exam is 10\% of your grade and your final exam can be used to replace your lowest midterm grade if it is higher than one of your midterms. The midterms are an individual assessment of your ability to apply the programming skills learned in lectures, recitation assignments, and homework assignments. If you do not show a skill level that is at or above this threshold, you will not receive a grade better than D+ in the course overall. A grade of C- in this class is required to take the next class in the computer science sequence. In addition, you will not get a passing score for the class by just showing up for the exams. You still need to turn in all other class work, including all projects. 

\begin{table}[H]
    \centering
    \begin{tabular}{c|c}
        Midterm 1 & Friday, Oct 4\\
        Midterm 2 & Friday, Oct 25 \\
        Midterm 3 & Friday, Nov 15
    \end{tabular}
\end{table}

\subsection{Grading Scale}

At the end of the course, letter grades will be assigned via the following formula:

\begin{table}[H]
    \centering
    \begin{tabular}{|c|c|}
    Letter Grade & Percentage Grade \\ \hline
    A & 93-100 \\
    A- & 90-92.99 \\
    B+ & 87-89.99 \\
    B & 83-86.99 \\
    B- & 80-82.99 \\
    C+ & 77-79.99 \\
    C & 73-76.99 \\
    C- & 70-72.99 \\
    D+ & 67-69.99 \\
    D & 63-66.99 \\
    D- & 60-62.99 \\
    F & <60 \\
    \end{tabular}
\end{table}

\section{Course Requirements}

\subsection{Method of Instruction}

Attendance at all class meetings is highly recommended, and attendance in recitations is required.

10\% of your grade is participation in lecture, which will include small in-lecture activities that you must be present to complete.

You are responsible for knowing the material presented during class, even if you were not in attendance when the material was presented. Previous experience has shown that students who do not attend class regularly often receive a failing grade and have to repeat the class the following semester.

Read the daily lecture material and watch pre-recorded topic videos before lecture. Lecture sessions will utilize discussion and activities that build on this content, along with the materials from previous days. Lectures will be recorded and posted to Canvas for later use within the course.

Recitation activities will include discussion and problem-solving, as well as provide a chance for addressing questions about course content. You will complete a Canvas quiz during recitation. 

Your TA(s) will take attendance for recitation each week. If you need to miss recitation, make arrangements with your TA(s), to attend a different recitation section for the week. You can find the list of TA’s and the times for each recitation on the start here page. If you miss recitation, you will not get credit for the Recitation assignment that week. 

Recitation will have similar expectations for lecture. Be professional, be on time, and do not be disruptive.

Good tips for maintaining a professional environment:

\begin{itemize}
    \item Decide before lecture if you are going to attend.
    \item If you attend, be professional, be on time, and do not be disruptive.
    \item Turn off your cell phone.
    \item Bring your laptop, but restrict its use to class activity.
    \item Put away newspapers and magazines.
    \item Refrain from disruptive conversation.
    \item In general, respect all interactions related to this class and the students that desire to partake.
\end{itemize}

\subsection{Required Texts and Other Materials}

\subsubsection{Textbook}

Brief C++ Late Objects, Third Edition, Cay S. Horstmann, Wiley. 2017. ISBN: 978-1118674260.

This book is available in digital format and print format.

It is presented in an Enhanced E-text format readable on computers, tablets, and Kindle devices through Canvas. All Enhanced E-text sections include many different forms of guidance to help students build confidence and tackle the task at hand, including Self Check and Practice activities along with end-of-section Review Exercises, Practice Exercises and Programming Projects.

Content in the new edition is much improved, and if you choose an older edition you will miss out on the interactive activities and exercises.

\subsubsection{Software}

We will be using VS Code as our primary Interactive Development Environment (IDE). This installation process is supported in the Getting Started and Week 1 materials of the course.

We will also be using Canvas as our Learning Management System (LMS). Although you only need to do minimal work directly on Canvas, you will need to know how to navigate Canvas to access course materials, see announcements, submit your homework, understand your grades, and view course notes and project descriptions.

You will also need to be expected to communicate and collaborate on Ed. You can post your questions on Ed in the appropriate thread for each week, each assignment and each problem. Your questions will be answered by another student or one of the members of the instructional team. Please also answer any questions that you see on Ed. As a group we are much more efficient than any one individual.

\subsubsection{Other Materials}

You must have a back-up system. Lost work, internet was broken, computer froze up, etc. are not valid reasons for missing due dates.

For this course it is highly recommended that you get a Dropbox account, a Google Drive account or invest in a USB memory stick, to save your files. Computer crashes absolutely WILL happen and you are responsible for saving and keeping backup copies of your work. Also, Google Drive and Dropbox keep a versioning system. If you accidentally delete your file, you will be able to recover it.

A limited amount of printing may be required in this class. You need to ensure that your printing account has sufficient funds for this. Your initial allocation may deplete quickly, depending on your other printing activities. If this causes problems, please contact the course staff at csci1300@colorado.edu .
    
\section{Collaboration}

\subsection{Collaboration, Plagiarism, and Honor Code}

Collaboration is highly recommended and a valuable skill to develop at this point in your life.

Collaboration is defined as working with other people to ask, and answer, questions about how coding is done. This can be accomplished verbally or with pseudo-code or with pictures or with flow charts or with discussions about what syntax means... Collaboration is NOT defined as showing someone your source code, or by cutting-and-pasting code, or by a group project where everyone talks but only one person codes, or by using a paid tutor to show you code, or by using internet sites to copy code.

Although you must work with other students to fully understand coding, you must never ever reveal your source code, or copy source code, or leave your browser open, or leave your computer unattended. Stolen code results in ALL involved parties being sent to honor council.

The Computer Science Department at the University of Colorado at Boulder encourages collaboration among students. You are encouraged to work with a partner for some assignments (Final Project), but you are also free to tackle the work alone.

Many forms of collaboration are acceptable and encouraged. In computer science courses, it is usually appropriate to ask others— TAs, LAs, instructor, or other students—for hints and debugging help or to talk generally about problem solving strategies and program structure. In fact, we strongly encourage you to seek such assistance when you need it. Discuss ideas together, but do the coding on your own .

\textbf{Rule 1: }You must not submit or look at solutions or program code that are not your own. Do not search for online solutions. This is not an appropriate way to “check your work,” “get a hint,” or “see alternative approaches.”

\textbf{Rule 2: }You must not share your solution code with other students, nor ask others to share their solutions with you. Do not leave copies of your work on public computers nor post your solution code on a public website.

\textbf{Rule 3: }You must indicate on your submission any assistance you received. It is fine to discuss ideas and strategies, but you should be careful to write your programs on your own.

\textbf{Be aware: }all submissions are subject to automated plagiarism detection. The entire point of the CU Honor Code is that we all benefit from working in an atmosphere of mutual trust. Do not take advantage of that trust. CU employs powerful automated plagiarism detection tools that compare assignment submissions with other submissions from the current and previous quarters. The tools also compare submissions against a wide variety of online solutions. These tools are effective at detecting unusual resemblances in programs, which are then further examined by the course staff. The staff then make the determination as to whether submissions are deemed to be potential infractions of the Honor Code.

To support students in collaboration the Department has created a Collaboration Policy that makes explicit when their collaborative behavior is within the bounds of the Collaboration Policy and when it is actually academic dishonesty, which would be considered a violation of the University’s Honor Code. All students of the University of Colorado at Boulder are responsible for knowing and adhering to the University’s Honor Code. Violations of this policy may also include cheating, plagiarism, academic dishonesty, fabrication, lying, bribery, and threatening behavior. Collaboration on homework assignments is allowed and encouraged. Students are most successful when they are working with other students to understand new concepts. The ultimate goal is that you fully understand the code you develop. 

Plagiarism includes using material from outside sources (e.g. Internet sources, chatGPT, other AI tools, chegg.com, coursehero.com, a tutor) without clear identification and citation. Unless otherwise specified, you may make use of outside resources (internet, other books, other people), but then you must give credit by citing your sources in the comments inside your code.

\subsubsection{Citing examples (assuming // indicates beginning of code comment):}

\begin{itemize}
    \item \mintinline{c++}{// Modified version from https://github.com/Phhere?MOSS-PHP}

    \item \mintinline{c++}{// Adapted from Program #7.2 in book “Accelerated C++” by Stroustrup}
    
    \item \mintinline{c++}{// Worked with Joe Smith from class to come up with algorithm for sorting}
    
    \item \mintinline{c++}{//Received suggestions from stackExchange website (see http://…)}
    
    \item \mintinline{c++}{//Worked with a tutor on the algorithm for the STORE function}
\end{itemize}

A good rule of thumb: “if it did not come from your brain, then you need to attribute where you got.”

\textbf{Note: }you do not need to cite if you are adapting from slides for the course or the required textbook for the course of from the hired staff for the course.

All homework assignments, all quizzes, all labs, and all exams will be required to be completed without outside resources . These will be clearly marked as individual assignments: the Canvas submission is individual. Use of outside resources would violate the collaboration policy.

\subsection{Adhering to the Collaboration Policy:}

Some examples of violating the collaboration policy include, but are not limited to:

\begin{itemize}
    \item Sharing a file (source code) with someone else.
    \item Submitting a file that someone else shared with you.
    \item Stealing a copy of someone else’s work and submitting as your own, even with modification.
    \item Copying outside resources.
    \item Using outside resources and not citing your sources.
    \item Posting on websites like chegg.com. coursehero.com or craigslist.org soliciting a solution to an (or part of an) assignment.
    \item Soliciting help with commenting your code also constitutes a violation of the collaboration policy.
    \item Copying solutions from website like chegg.com, coursehero.com, and other websites.
    \item Using a solution the “tutor” gave you.
\end{itemize}

Examples of collaborating correctly :

\begin{itemize}
    \item Sharing pseudo-code.
    \item Asking another student for a helpful suggestion. Verbally, not written.
    \item Reviewing another student’s code for bugs/errors.
    \item Working together on the whiteboard, or paper, to figure out how to approach and solve the problem. In this case you must include that person’s name in your collaboration list at the top of your submission. This includes working with a tutor.
\end{itemize}

One way to know you are collaborating well is if everyone is developing the code solution individually. This collaboration policy requires that you be able to create the code, or solve the problem on your own before you submit your assignment. You can brainstorm a solution or algorithm with a friend, but the submitted code should not be the same code.

Even if collaboration is stated, it is a violation of the Honor Code to submit effectively identical code with another student or an outside source. 

If two or more students submit the same solution, claiming they have been “working with the same tutor”, that constitutes a violation of the Honor Code.

Any discovered incidents of violation of this collaboration policy will be treated as violations of the University’s Academic Integrity Policy and will lead to an automatic academic sanction in the course and a report to both the College of Engineering and Applied Science and the Honor Code Council. The academic sanction will be an automatic F.

Note: The instructor reserves the right to change the policy and to apply different academic sanctions if the violation justifies it. Students who are found to be in violation of the Academic Integrity Policy can be subject to non-academic sanctions as well, including but not limited to university probation, suspension, or expulsion.

Collaboration boundaries are hard to define crisply, and may differ from class to class. If you are in any doubt about where they lie for a particular course, it is your responsibility to ask the course instructor.

Students that leave their computers unprotected are also subject to course sanctions mentioned above. When similar code is found, ALL parties are subject to sanctions. This includes the person whose code was “taken”. Protect your source code at all times. Do not forget to sign out of public computers or leave your own machine unattended.

\section{Communication}

\textbf{Please send all general course questions to: csci1300@colorado.edu}

As a member of the CU community you are expected to consistently demonstrate integrity and honor through your everyday actions.

\subsection{Professional Email Expectations}

Any email correspondence related to the class should be sent from a colorado.edu email address. Please note that we do not read email between 5pm and 9am, or during the weekends. You can expect a response within 24 - 48 hours during the week and within 48 - 72 hours if sent on the weekend.

Send email messages to faculty and staff using a professional format.

Tips for a professional email include:

\begin{itemize}
    \item Always fill in the subject line with a topic that indicates the reason for your email to your reader.
    \item Respectfully address the individual to whom you are sending the email (e.g., Dear Professor Smith).
    \item Avoid email or text message abbreviations.
    \item Be brief and polite.
    \item Add a signature block with appropriate contact information.
    \item Reply to email messages with the previously sent message. This will allow your reader to quickly recall the questions and previous conversation.
\end{itemize}
    

\subsection{Office Hours}

Faculty, TAs, and staff members are very willing to assist with your academic and personal needs. However, multiple professional obligations make it necessary for us to schedule our availability.

Suggestions specific to interactions with faculty and staff include:

\begin{itemize}
    \item Respect posted office hours.
        Plan your weekly schedule to align with scheduled office hours. There is a variety of availability throughout the week.
    \item Avoid disrupting ongoing meetings within faculty and staff offices.
        Please wait until the meeting concludes before seeking assistance.
\end{itemize}
    

Our office hours are structured as group meetings.

    Group office hours are on Monday through Friday, where multiple students can join in for help, and one or more members of the instructional team will be available to answer questions.

Please check out our Office Hours page on Canvas for up to date availability.

\subsubsection{What are Office Hours?}

    It is much like a classroom. It is a place to ask questions, explore further, discussion strategies, explore related topics and support your classmates and contribute to the class. Office hours are optional.

\subsubsection{How do we use Office Hours?}

\begin{itemize}
    \item Discuss class content directly with the instructor and other students.
    \item Practice professional collaboration strategies.
    \item Instructors may use breakout groups or/and guide discussions, at their discretion.
    \item Instructors may decide which topics will be discussed based on what will optimize learning.
    \item Please wear appropriate attire and be aware of your environment.
\end{itemize}
    
\subsubsection{What are Office Hours not?}

\begin{itemize}
    \item Answer forum.
    \item Tutoring session.
    \item Instructors cannot help you completely debug your code.
    \item Instructor may not reply to all inquiries and let other students speak as appropriate.
\end{itemize}
    

\subsection{Ed}

We will use Ed as the center of communication for this course. Whenever you have content related questions about the course, especially ones that you believe other students may also have, you should post these questions on Ed. You can find the link on Canvas. 




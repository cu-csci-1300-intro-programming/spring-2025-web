\chapter*{CSCI 1300 Syllabus Summer 2024}
\addcontentsline{toc}{chapter}{CSCI 1300 Syllabus Summer 2024}

This course teaches techniques for writing computer programs in higher level programming languages to solve problems of interest in a range of application domains. Appropriate for students with little to no experience in computing or programming.

\section{Objectives}
This class covers the basics of computer programming using C++. The focus will be on understanding the basic components of computer languages. Upon completion of the course, the student can:

\begin{itemize}
    \item Design and construct algorithms for problem solving by applying processes of abstraction and program decomposition.
    \item Implement fundamental programming constructs, such as variables, expressions, conditionals, and iterative control structures, in a higher-level language.
    \item Evaluate and implement simple I/O, such as user input and file I/O, including the necessity for external input to a program and the role of external data storage.
    \item Design and explain the role of functions in program construction, including an understanding of parameter passing and return values.
    \item Describe the properties of data types, including the primitive types of numbers, characters, and booleans, as well as more complex data types, such as arrays, records, and strings.
    \item Use an Integrated Development Environment to produce code that is free of syntactical, logical, and run-time errors. Understand the process of debugging as part of software development.
    \item Design and create code using object-oriented design methodology, including an understanding of objects and classes, information encapsulation, and efficient class design.
    \item Use third-party code to accomplish a programming objective. This includes learning to read code written by another individual and modifying the code, or using third-party libraries.
    \item Develop an understanding that software development is a dynamic, social process, and that learning how to seek out information is a necessary skill for success.
\end{itemize}

\section{Course Schedule}
Here is an approximate course schedule. This schedule is subject to changes as the semester progresses; an up-to-date copy is available on Canvas.

\begin{longtable}[H]{|p{1.5in}|p{1in}|p{2in}|p{1.5in}|} \hline
        Week & Date & Topics & Assignments \\ \hline
Week 1: Intro & Mon, June 3 & Syllabus + Intro to Computing &  \\
& Tues, June 4 & Variables and Arithmetic Operators & Midnight: Warmup/Prelab \\
& Wed, June 5 & Recitation & Midnight: Recitation \\
& Thurs, June 6 & User I/O & \\
& Fri, June 7 & Pseudocode and Algorithms & Midnight: Homework \\
& Sat, June 8 & & \\
& Sun, June 9 & & \\ \hline
Week 2: Functions and Switches & Mon, June 10 & Characters and Switches & \\
 & Tues, June 11 & Review Switches, Relational/Logic Operators, If/Else & Midnight: Warmup/Prelab \\
 & Wed, June 12 & Recitation & Midnight: Recitation  \\
 & Thurs, June 13 & Boolean Logic, If/ElseIf/Else  & \\
 & Fri, June 14 & Nested If/Else & Midnight: Homework \\
 & Sat, June 15 & & \\
 & Sun, June 16 & & \\ \hline
Week 3: Conditionals & Mon, June 17 & Intro to Functions & \\
 & Tues, June 18 & Strings & Midnight: Warmup/Prelab \\
 & Wed, June 19 & NO CLASSES & NO CLASS \\
 & Thurs, June 20 & Functions, Variables, and Strings – Testing  &  \\
 & Fri, June 21 & Exam 1 & Midnight: Homework \\
 & Sat, June 22 & & \\
 & Sun, June 23 & & \\ \hline
Week 4: Loops and Arrays & Mon, June 24 & While Loops & \\
 & Tues, June 25 & For Loops & Midnight: Warmup/Prelab \\
 & Wed, June 26 & Recitation & Midnight: Recitation \\
 & Thurs, June 27 & Arrays & \\
 & Fri, June 28 & Pass by Reference/Value, File I/O & Midnight: Homework \\
 & Sat, June 29 & & \\
 & Sun, June 30 & & \\ \hline
Week 5: Objects & Mon, July 1 & Arrays/File IO pt 3 & \\
 & Tues, July 2 & Classes/Objects (Intro) & Midnight: Warmup/Prelab \\
 & Wed, July 3  & Recitation & Midnight: Recitation \\
 & Thurs, July 4 & NO CLASS  & \\
 & Fri, July 5 & Classes/Objects (Continued) & \\
 & Sat, July 6 & & \\
 & Sun, July 7 & & Midnight: Project 1, Part 1 \\ \hline
Week 6: Objects & Mon, July 8 & Objects & \\
 & Tues, July 9 & Objects & Midnight: Project 1, Part 2 \\
 & Wed, July 10 & Recitation & Midnight: Recitation \\
 & Thurs, July 11 & Objects & Midnight: Project 2 Proposal \\
 & Fri, July 12 & Exam 2 & Midnight: Homework \\
 & Sat, July 13 & & \\
 & Sun, July 14 & & \\ \hline
Week 7: Vectors, Structs, Randomness and Recursion & Mon, July 15 & Vectors & \\
 & Tues, July 16 & Structs/Randomness & Midnight: Warmup/Prelab \\
 & Wed, July 17 & Recitation & Midnight: Recitation \\
 & Thurs, July 18 & Recursion & \\
 & Fri, July 19 & Recursion  & \\
 & Sat, July 20 & & \\
 & Sun, July 21 & & Midnight: Project 2 \\ \hline
Week 8: Special Topics & Mon, July 22 & Python & \\
 & Tues, July 23 & Sorting and Computational Complexity & Midnight: Warmup/Prelab \\
 & Wed, July 24 & Recitation & Midnight: Recitation \\
 & Thurs, July 25 & Project Showcase  & \\
 & Fri, July 26 & Final Exam (Exam Retakes) & \\ \hline
\end{longtable}

\section{Assignments and Grading}
\subsection{Weekly Assignments}

For each week, there will be a PDF file available that has four sections: one will be a summary of the topics and materials covered, and then there will be a Warmup/Prelab section (15\% of weekly grade), a Recitation section (35\% of weekly grade), and a Homework section (50\% of weekly grade). The Warmup section will be a series of questions to get you adjusted to the new content. The Warmup section will be due at midnight on Tuesday nights. Recitations will then address these questions and provide the correct answers from the warmup, and then in recitation you will answer the recitation questions.  The recitation questions will be due at midnight on Wednesdays. Finally, the homework will be due on Friday at midnight. The warmup, recitation, and homework will build on each other and the questions will be designed to increase in depth and difficulty throughout the week. On weeks where projects are due, the projects will replace homework. 

\subsection{Exams}

There will be two exams during the course. The exams will be the only part of the course where you do not have access to external resources or collaboration; however, you will be given a print out of the syntax guide with your test. The goal of these tests is not to test your precision offline, only your general understanding and mastery of concepts. If you do poorly on an exam, you will have a chance to retake it during the final exam period at the end of the class. There will be no additional final exam.

\subsection{Projects}

There will be two projects during the course. These projects will take the place of a homework during their respective week. They will count for more of your grade, and you will have more time to do them.  Projects will be graded for correctness as well as style (legibility, variable names, comments, conciseness). The style grade will be the homework grade for that week. There will be interview grading for the projects, which will serve as a chance to earn back up to half the points missed for correctness.

\subsection{Points assignment:}
\begin{table}[H]
    \centering
    \begin{tabular}{|c|c|} \hline
         Assignment 	& Percent of final grade \\ \hline
        Weekly Assignments & 50\% \\
        Exams & 20\% \\
        Project 1 & 10\% \\
        Final Project & 20\% \\ \hline
    \end{tabular}
\end{table}

\section{Attendance}

Attendance is strongly encouraged. If you miss a day, you will need to contact me with the reason -- communication is key. In the recovery stages of the pandemic, please take sick days as you need them. I also believe in mental health days. However, many absences or no communication about attendance may result in: being administratively dropped from the course, poor grades, or reporting low attendance to the university. If attendance is a struggle for you, please communicate with me and we will discuss what is necessary on a case-by-case basis.

\section{Texts and Materials}

The textbook I would recommend is:

Brief C++ Late Objects, Third Edition, Cay S. Horstmann, Wiley. 2017. ISBN: 978-1118674260.

This book is available only in digital format. This book is very helpful, and I would encourage you to get it, but it is not required as we will not be using any assignments from the textbook.

All lecture slides will be provided, and brief summaries of critical material will be available in the weekly assignment PDFs. 

For C++ syntax and usage, please refer to the C++ reference guide: https://cplusplus.com/reference/

\section{Collaboration, Plagiarism, and the Honor Code}

Collaboration is highly recommended and a valuable skill to develop at this point in your life.

Collaboration is defined as working with other people to ask, and answer, questions about how coding is done. This can be accomplished verbally or with pseudo-code or with pictures or with flow charts or with discussions about what syntax means... Collaboration is NOT defined as showing someone your source code, or by cutting-and-pasting code, or by a group project where everyone talks but only one person codes, or by using a paid tutor to show you code, or by using internet sites to copy code.

Although you must work with other students to fully understand coding, you must never ever reveal your source code, or copy source code, or leave your browser open, or leave your computer unattended. Stolen code results in ALL involved parties being sent to honor council.

The Computer Science Department at the University of Colorado at Boulder encourages collaboration among students. You are encouraged to work with a partner for some assignments (Final Project), but you are also free to tackle the work alone.

Many forms of collaboration are acceptable and encouraged. In computer science courses, it is usually appropriate to ask others— TAs, LAs, instructor, or other students—for hints and debugging help or to talk generally about problem solving strategies and program structure. In fact, we strongly encourage you to seek such assistance when you need it. Discuss ideas together, but do the coding on your own .

\textbf{Rule 1}: You must not submit or look at solutions or program code that are not your own. Do not search for online solutions. This is not an appropriate way to “check your work,” “get a hint,” or “see alternative approaches.”

\textbf{Rule 2}: You must not share your solution code with other students, nor ask others to share their solutions with you. Do not leave copies of your work on public computers nor post your solution code on a public website.

\textbf{Rule 3}: You must indicate on your submission any assistance you received. It is fine to discuss ideas and strategies, but you should be careful to write your programs on your own.

\textbf{Be aware}: all submissions are subject to automated plagiarism detection. The entire point of the CU Honor Code is that we all benefit from working in an atmosphere of mutual trust. Do not take advantage of that trust. CU employs powerful automated plagiarism detection tools that compare assignment submissions with other submissions from the current and previous quarters. The tools also compare submissions against a wide variety of online solutions. These tools are effective at detecting unusual resemblances in programs, which are then further examined by the course staff. The staff then make the determination as to whether submissions are deemed to be potential infractions of the Honor Code.

To support students in collaboration the Department has created a Collaboration Policy that makes explicit when their collaborative behavior is within the bounds of the Collaboration Policy and when it is actually academic dishonesty, which would be considered a violation of the University’s Honor Code. All students of the University of Colorado at Boulder are responsible for knowing and adhering to the University’s Honor Code. Violations of this policy may also include cheating, plagiarism, academic dishonesty, fabrication, lying, bribery, and threatening behavior. Collaboration on homework assignments is allowed and encouraged. Students are most successful when they are working with other students to understand new concepts. The ultimate goal is that you fully understand the code you develop. 

Plagiarism includes using material from outside sources (e.g. Internet sources, chegg.com, coursehero.com, a tutor) without clear identification and citation. Unless otherwise specified, you may make use of outside resources (internet, other books, other people), but then you must give credit by citing your sources in the comments inside your code.

Citing examples (assuming // indicates beginning of code comment):

\begin{minted}{c++}
    // Modified version from https://github.com/Phhere?MOSS-PHP
    
    // Adapted from Program #7.2 in book “Accelerated C++” by Stroustrup
    
    // Worked with Joe Smith from class to come up with algorithm for sorting
    
    //Received suggestions from stackExchange website (see http://…)
    
    //Worked with a tutor on the algorithm for the STORE function
\end{minted}

A good rule of thumb: “if it did not come from your brain, then you need to attribute where you got.”

Note: you do not need to cite if you are adapting from slides for the course or the required textbook for the course of from the hired staff for the course.

All homework assignments, all quizzes, all labs, and all exams will be required to be completed without outside resources . These will be clearly marked as individual assignments: the Canvas submission is individual. Use of outside resources would violate the collaboration policy.

\subsection{Adhering to the Collaboration Policy:}

Some examples of violating the collaboration policy include, but are not limited to:

\begin{itemize}
    \item Sharing a file (source code) with someone else.
    \item Submitting a file that someone else shared with you.
    \item Stealing a copy of someone else’s work and submitting as your own, even with modification.
    \item Copying outside resources.
    \item Using outside resources and not citing your sources.
    \item Posting on websites like chegg.com. coursehero.com or craigslist.org soliciting a solution to an (or part of an) assignment.
    \item Soliciting help with commenting your code also constitutes a violation of the collaboration policy.
    \item Copying solutions from website like chegg.com, coursehero.com, and other websites.
    \item Using a solution the “tutor” gave you.
\end{itemize}

Examples of collaborating correctly :

\begin{itemize}
    \item Sharing pseudo-code.
    \item Asking another student for a helpful suggestion. Verbally, not written.
    \item Reviewing another student’s code for bugs/errors.
    \item Working together on the whiteboard, or paper, to figure out how to approach and solve the problem. In this case you must include that person’s name in your collaboration list at the top of your submission. This includes working with a tutor.
\end{itemize}

One way to know you are collaborating well is if everyone is developing the code solution individually. This collaboration policy requires that you be able to create the code, or solve the problem on your own before you submit your assignment. You can brainstorm a solution or algorithm with a friend, but the submitted code should not be the same code.

Even if collaboration is stated, it is a violation of the Honor Code to submit effectively identical code with another student or an outside source. 

If two or more students submit the same solution, claiming they have been “working with the same tutor”, that constitutes a violation of the Honor Code.

Any discovered incidents of violation of this collaboration policy will be treated as violations of the University’s Academic Integrity Policy and will lead to an automatic academic sanction in the course and a report to both the College of Engineering and Applied Science and the Honor Code Council. The academic sanction will be an automatic F.

Note: The instructor reserves the right to change the policy and to apply different academic sanctions if the violation justifies it. Students who are found to be in violation of the Academic Integrity Policy can be subject to non-academic sanctions as well, including but not limited to university probation, suspension, or expulsion.

Collaboration boundaries are hard to define crisply, and may differ from class to class. If you are in any doubt about where they lie for a particular course, it is your responsibility to ask the course instructor.

Students that leave their computers unprotected are also subject to course sanctions mentioned above. When similar code is found, ALL parties are subject to sanctions. This includes the person whose code was “taken”. Protect your source code at all times. Do not forget to sign out of public computers or leave your own machine unattended.

\section{Communication}

For questions on content, homework, assignments, etc., please use the course Ed page. You can sign up here: 

You can also attend office hours in CSEL. The office hours schedule is available under the course pages on Canvas.

For administrative questions or notes (absences, grade questions, etc.), please email me directly at jordan.tate@colorado.edu and I will respond as promptly as possible. However, I will not answer emails outside of work hours, so please allow at least one business day for a response. This means that waiting until the last moment to email will not get you a reply, so please plan and prepare accordingly.

\section{Course Policies}

\subsection{Requirements for COVID-19}

As a matter of public health and safety, all members of the CU Boulder community and all visitors to campus must follow university, department and building requirements and all public health orders in place to reduce the risk of spreading infectious disease. The CU Boulder campus is currently mask-optional. However, if public health conditions change and masks are again required in classrooms, students who fail to adhere to masking requirements will be asked to leave class, and students who do not leave class when asked or who refuse to comply with these requirements will be referred to Student Conduct and Conflict Resolution. For more information, see the policy on classroom behavior and the Student Code of Conduct. If you require accommodation because a disability prevents you from fulfilling safety measures related to infectious disease, please follow the steps in the “Accommodation for Disabilities” statement on this syllabus.

If you feel ill and think you might have COVID-19 or if you have tested positive for COVID-19, you should stay home and follow the further guidance of the Public Health Office regarding how long to stay in isolation. If you have been in close contact with someone who has COVID-19 but do not have any symptoms and have not tested positive for COVID-19, you do not need to stay home. In this class we will do our best to accommodate any absences due to illness, but you must contact me via email.

\subsection{Classroom Behavior}

Both students and faculty are responsible for maintaining an appropriate learning environment in all instructional settings, whether in person, remote or online. Those who fail to adhere to such behavioral standards may be subject to discipline. Professional courtesy and sensitivity are especially important with respect to individuals and topics dealing with race, color, national origin, sex, pregnancy, age, disability, creed, religion, sexual orientation, gender identity, gender expression, veteran status, political affiliation or political philosophy.  For more information, see the classroom behavior policy, the Student Code of Conduct, and the Office of Institutional Equity and Compliance.

\subsection{Preferred names and pronouns}

CU Boulder recognizes that students' legal information doesn't always align with how they identify. Students may update their preferred names and pronouns via the student portal; those preferred names and pronouns are listed on instructors' class rosters. In the absence of such updates, the name that appears on the class roster is the student's legal name.

If you want to talk with me about any of these topics (do you want to try a new name or pronouns, but don't want to update them officially? Do you prefer something different in personal communications vs in front of class?) please feel free -- I want to make sure you are comfortable in your class environment.

\subsection{Honor Code}

All students enrolled in a University of Colorado Boulder course are responsible for knowing and adhering to the Honor Code. Violations of the Honor Code may include, but are not limited to: plagiarism, cheating, fabrication, lying, bribery, threat, unauthorized access to academic materials, clicker fraud, submitting the same or similar work in more than one course without permission from all course instructors involved, and aiding academic dishonesty. All incidents of academic misconduct will be reported to Student Conduct \& Conflict Resolution (honor@colorado.edu); 303-492-5550). Students found responsible for violating the Honor Code will be assigned resolution outcomes from the Student Conduct \& Conflict Resolution as well as be subject to academic sanctions from the faculty member. Additional information regarding the Honor Code academic integrity policy can be found on the Honor Code website.

\subsection{Sexual Misconduct, Discrimination, Harassment and/or Related Retaliation}

CU Boulder is committed to fostering an inclusive and welcoming learning, working, and living environment. University policy prohibits sexual misconduct (harassment, exploitation, and assault), intimate partner violence (dating or domestic violence), stalking, protected-class discrimination and harassment, and related retaliation by or against members of our community on- and off-campus. These behaviors harm individuals and our community. The Office of Institutional Equity and Compliance (OIEC) addresses these policies, and individuals who believe they have been subjected to misconduct can contact OIEC at 303-492-2127 or email cureport@colorado.edu. Information about university policies, reporting options, and support resources can be found on the OIEC website.

Please know that faculty and graduate instructors have a responsibility to inform OIEC when they are made aware of any issues related to these policies regardless of when or where they occurred to ensure that individuals impacted receive information about their rights, support resources, and resolution options. To learn more about reporting and support options for a variety of concerns, visit Don’t Ignore It.
Religious Holidays

Campus policy regarding religious observances requires that faculty make every effort to deal reasonably and fairly with all students who, because of religious obligations, have conflicts with scheduled exams, assignments or required attendance. We will make every effort to accommodate your religious obligations provided that you notify me via writing well in advance of the scheduled conflict. Whenever possible, you should notify us at least two weeks in advance of the conflict to request special accommodations. No exceptions can be made after the scheduled event has passed. See the campus policy regarding religious observances for full details.

\subsection{Canvas Privacy Policy}

You can find a copy of the Canvas Privacy Policy on the Instructure Product Privacy Policy page.

\subsection{Canvas Accessibility Statement}

You can find a copy of the Canvas Accessibility Statement on the Accessibility within Canvas page.
\chapter*{Week 13: Vectors}
\addcontentsline{toc}{chapter}{Week 13: Vectors}
\setcounter{chapter}{14}
\setcounter{section}{0}

\begin{abstract}
This week you will:
\begin{enumerate}
    \item Learn what vectors are
    \item Practice using vectors 

\end{enumerate}
    
\end{abstract}

\section{Background}

\subsection{Vectors}

Let's start with something we already know about - Arrays.

To recap, an array is a contiguous series that holds a fixed number of values of the same datatype.

A vector is a template class that uses all of the syntax that we used with vanilla arrays, but adds in functionality that relieves us of the burden of keeping track of memory allocation for the arrays. It also introduces a bunch of other features that makes handling arrays much simpler.

First things first. We need to include the appropriate header files to use the vector class.

\mintinline{c++}{#include <vector>}

We can now move on to declaring a vector. This is general format of any vector declaration:

\mintinline{c++}{vector <datatype_here> name(size);}

The size field is optional. Vectors are dynamically-sized, so the size that you give them during initialization isn't permanent - they can be resized as necessary.

You can access elements of a vector in the same way you would access elements in an array, for example array[4]. Remember, indices begin from 0.

The C++ vector class comes with \textcolor{cyan}{\href{https://www.cplusplus.com/reference/vector/vector/vector/}{several member functions}} available in the C++ reference guide, but following are the ones you will need in this week:

\begin{itemize}
    \item \mintinline{c++}{size()} return the size of a vector
    \item \mintinline{c++}{at()} takes an integer parameter for index and returns the value at that position
\end{itemize}

Adding elements to the vector is done primarily using two functions

\mintinline{c++}{push_back()} takes in one parameter (the element to be added) and appends it to the end of the vector. Here is an example:

\begin{example}
    How to use \mintinline{c++}{push_back()} with vectors:

    \begin{minted}{c++}
vector <int> vector1; // initializes an empty vector
vector1.push_back(1); //Adds 1 to the end of the vector. 
vector1.push_back(3); //Adds 3 to the end of the vector. 
vector1.push_back(4); //Adds 4 to the end of the vector. 
cout<< vector1.size(); //This will print the size of the vector - in this case, 3.
// vector1 looks like this: [1, 3, 4]
    \end{minted}
\end{example}

\mintinline{c++}{insert()} can add an element at some position in the middle of the vector.

\begin{example}
    How to use \mintinline{c++}{insert} with vectors:

    \begin{minted}{c++}
// vectorName.insert(vectorName.begin() + position, element)
vector1.insert(vector1.begin() + 1, 2);
cout << vector1.at(1) << endl; // 2 is at index=1
// vector1 looks like this: [1, 2, 3, 4]
    \end{minted}
\end{example}

Here, the \mintinline{c++}{begin} function returns an iterator to the first element of the vector. Due to the nature of an iterator, this allows for the utility of using \mintinline{c++}{begin()} to refer to the first element and \mintinline{c++}{begin()+k} would refer to the kth element in the vector, starting at 0.

Elements can also be removed.

\mintinline{c++}{pop_back()} deletes the last element in the vector.

\begin{example}
    How to use \mintinline{c++}{pop_back()} with vectors:
    \begin{minted}{c++}
vector <int> vector1; // initializes an empty vector
vector1.push_back(1); //Adds 1 to the end of the vector. 
vector1.push_back(3); //Adds 3 to the end of the vector. 
vector1.push_back(4); //Adds 4 to the end of the vector. 
vector1.pop_back(); //Removes the last element of the vector.
//vector1 looks like this: [1, 3]
    \end{minted}
\end{example}

\mintinline{c++}{erase()} can delete a single element at some position, which is shown below using the iterator function of \mintinline{c++}{begin()} to erase the first element of the vector.

\begin{example}
    How to use \mintinline{c++}{erase()} with vectors:

    \begin{minted}{c++}
// vector_name.erase(vector_name.begin() + position)
vector1.erase(vector1.begin() + 0);
cout << vector1.at(0) << endl; //2 is at index=0
// vector1 looks like this: [2, 3, 4]
    \end{minted}
\end{example}

It may be useful to think of vectors relationship to arrays as something similar to strings vs arrays of characters; they are similar concepts, but with added utility and flexibility that is helpful. Vectors are also passed by value (like strings) instead of passed by reference (like arrays). This might look something like:

\begin{example}
    Full vector example:

    \begin{minted}{c++}
void myVecEditFunction(vector <int> vec){
   vec.erase(vec.begin());
   //vec now contains the original vector minus the starting element
}

...

int main(){
   vector <int> originalVector = {1, 2, 3};
   myVecEditFunction(originalVector);
   //originalVector still looks like [1, 2, 3]
}
    \end{minted}
\end{example}

\subsection{Randomness}

Random numbers are a valuable tool for a number of applications, including writing games where we want random chance to be a factor. There are limitations in being able to make a truly random number generator with code, but we have tools to get close enough.

\mintinline{c++}{rand()} function is an inbuilt function in C++ STL, which is defined in header file \mintinline{c++}{<cstdlib>}. \mintinline{c++}{rand()} is used to generate a series of random numbers. The random number is generated by using an algorithm that gives a series of non-related numbers whenever this function is called. The \mintinline{c++}{rand()} function is used in C++ to generate random numbers in the range \mintinline{c++}{[0, RAND_MAX)}.

\mintinline{c++}{RAND_MAX}: It is a constant whose default value may vary between implementations but it is granted to be at least 32767.

The syntax for the function is: \mintinline{c++}{int rand(void);} where \mintinline{c++}{int} is the return type and the parameter list is void (i.e. needs to parameters).

However in order to ensure that the random sequence of numbers is unique each time, we must choose a unique starting seed for the random generator.

\mintinline{c++}{srand()} function is an inbuilt function in C++ STL,  which is defined in \mintinline{c++}{<cstdlib>} header file. \mintinline{c++}{srand()} is used to initialize random number generators. The \mintinline{c++}{srand()} function sets the starting point for producing a series of pseudo-random integers. If \mintinline{c++}{srand()} is not called, the \mintinline{c++}{rand()} seed is set as if \mintinline{c++}{srand(1)} were called at the program start. Any other value for seed sets the generator to a different starting point. 

Here are the two function prototypes for \mintinline{c++}{srand()} to see the syntax:

\begin{minted}{c++}
void srand( unsigned seed );
int srand( unsigned int seed);
\end{minted}

Seeds the pseudo-random number generator used by rand() with the value seed.
Parameter \textbf{seed}: A seed for a new sequence of pseudo-random numbers to be returned by successive calls to rand()

Return value: This function returns a pseudo-generated random number.

\textbf{Note:} The pseudo-random number generator should only be seeded once, before any calls to rand(), and at the start of the program. It should not be repeatedly seeded or reseeded every time you wish to generate a new batch of pseudo-random numbers. 

Standard practice is to use the result of a call to \mintinline{c++}{srand(time(0))} as the seed. However, \mintinline{c++}{time()} returns a \mintinline{c++}{time_t} value which varies every time and hence the pseudo-random number varies for every program call. 

Here are a few examples of using randomness.

\begin{example}
    A short example to roll a die:

    \begin{minted}{c++}
int main(){
    //declare variables
    int dieRoll;
    //random seed 
    srand(time(0));

    dieRoll = rand()%6; //randomly generate a number 0 through 5
    dieRoll+=1; //add 1 to make it now store a number 1 through 6

    //printing a random number stored in a variable
    cout << "Our dice roll is " << dieRoll << endl; 

    //printing the random number directly
    cout << "Our D20 rolled " << rand()%20+1 << endl; 
}
    \end{minted}
\end{example}

\begin{example}
    A long example to play Rock, Paper, Scissors:

    \begin{minted}{c++}
int main(){
    //declare variables
    char userChoice; //to store the user's choice of R, P or S
    int compChoice; //to store the computer's choice
    srand(time(0));

    //ask the user for their choice
    cout << "Rock (R), Paper (P), or Scissors (S)?" << endl;
    cin >> userChoice;

    compChoice = rand()%3; //randomly choose a number 0, 1 or 2

    //Arbitarily choosing that 0 = R, 1 = P, 2 = S for comparison
    switch(userChoice){
        case 'R':
            switch(compChoice){
                case 0: 
                    cout << "Both chose rock-- tie!" << endl;
                    break;
                case 1:
                    cout << "You chose rock, the computer chose ";
                    cout << "paper -- you lose." << endl;
                    break;
                case 2:
                    cout << "You chose rock, the computer chose ";
                    cout << "scissors -- you win!" << endl;
            }
            break;
        case 'P':
            switch(compChoice){
                case 0: 
                    cout << "You chose paper, the computer chose ";
                    cout << "rock -- you win!" << endl;
                    break;
                case 1:
                    cout << "Both chose paper -- tie!" << endl;
                    break;
                case 2:
                    cout << "You chose paper, the computer chose ";
                    cout << "scissors -- you lose." << endl;
            }
            break;
        case 'S':
            switch(compChoice){
                case 0: 
                    cout << "You chose scissors, the computer "; 
                    cout << "chose rock -- you lose." << endl;
                    break;
                case 1:
                    cout << "You chose scissors, the computer "; 
                    cout << "chose paper -- you win!" << endl;
                    break;
                case 2:
                    cout << "Both chose scissors -- tie!" << endl;
            }
            break;
        default:
            cout << "Invalid choice." << endl;
    }

}

    \end{minted}
\end{example}

\subsection{Function Parameters: Pass By Reference}

In C++, you can pass parameters to functions either by value or by reference. Passing by reference allows you to modify the original variable within the function.

The syntax is:

\begin{minted}
void functionName(<data_type> &parameter_name)
{
    // Function body
}
\end{minted}

Example:

\begin{minted}{c++}
#include <iostream>
using namespace std;

// Function to increment a number by value
void incrementByValue(int num)
{
    num++;
}

// Function to increment a number by reference
void incrementByReference(int &num)
{
    num++;
}

int main()
{
    int x = 5;

    cout << "Before increment: " << x << endl;
    incrementByValue(x); // passing x by value
    cout << "After increment: " << x << endl;

    cout << "---" << endl;

    cout << "Before increment: " << x << endl;
    incrementByReference(x); // passing x by reference
    cout << "After increment: " << x << endl;

    return 0;
}
\end{minted}

\begin{sample}
Before increment: 5

After increment: 5

---

Before increment: 5

After increment: 6

\end{sample}

Key points to remember:

\begin{itemize}
    \item Parameters passed by reference are indicated by \& in the function declaration.
    \item Changes made to the reference parameter within the function affect the original variable.
    \item Pass by reference avoids unnecessary copying of the argument to the parameter variable, which can be more efficient.
    \item Arrays are passed by reference by default (so there is no preceding \& for an array parameter).
    \item References cannot be reassigned to refer to a different variable once initialized.
\end{itemize}

\section{PreQuiz}
\begin{problem}
True or False: Header files in C++ define the implementation details of functions and classes.
\end{problem}

\begin{problem}
True or False: When compiling multiple C++ files, header files (.h) are included in the compilation command.
\end{problem}

\begin{problem}
True or False: Source files in C++ use the .cpp extension and implement the class defined in the corresponding header file.
\end{problem}

\begin{problem}
Short Answer: Explain the purpose of using both header files (.h) and source files (.cpp) in organizing C++ programs. How do they contribute to code clarity and maintenance in larger projects?
\end{problem}

\begin{problem}
Fill in the blanks for this code which defines a State class which has both a default and parameterized constructor:

\begin{minted}{c++}
#include <iostream>
#include <string>
using namespace std;

class State {
public:
    State() {
        ______ = "Unknown";
        ______ = 0;
        ______ = 0.0;
    }

    State(string stateName, int statePopulation, double stateArea) {
        ______ = stateName;
        ______ = statePopulation;
        ______ = stateArea;
    }

    void displayInfo() {
        cout << "State: " << ______ << endl;
        cout << "Population: " << ______ << endl;
        cout << "Area: " << ______ << " sq miles" << endl;
    }

private:
    string name;
    int population;
    double area;
};

int main() {
    ______ defaultState;
    defaultState.________();

    ______ customState("California", 39538223, 163696);
    customState.________();

    return 0;
}
\end{minted}
\end{problem}

\begin{problem}
    Fill in the blanks for this code which is based on the last problem and has been changed to use both a header and source file.

Here is the header file:
\begin{minted}{c++}
#include <string>
using namespace std;

class State {
public:
    // Default constructor
    ______ ();

    // Non-default constructor
    ______ (string stateName, int statePopulation, double stateArea);

    void ______ ();  // Declare the method to display state information

private:
    string ______;     // Variable to hold the state's name
    int ______;        // Variable to hold the state's population
    double ______;     // Variable to hold the state's area
};
\end{minted}

Here is the corresponding source file:
\begin{minted}{c++}
// State.cpp
#include "State.h"
#include <iostream>
using namespace std;

// Default constructor implementation
State::______() {
    name = "Unknown";
    population = 0;
    area = 0.0;
}

// Non-default constructor implementation
State::______(string stateName, int statePopulation, double stateArea) {
    name = ______;
    population = ______;
    area = ______;
}

// displayInfo method implementation
void State::______() {
    cout << "State: " << ______ << endl;
    cout << "Population: " << ______ << endl;
    cout << "Area: " << ______ << " sq miles" << endl;
}
\end{minted}
\end{problem}

\section{Recitation}
%Map Class from the final project, Code Skeleton work

\subsection{The Map Class}
%TO-DO: Make this consistent; is it a board or a map?
%For the board files, edit them to be more clear about what the students should change and what they should not. Provide the Tile struct and update it to use a string instead of a char. Add lots of comments!
%List which files they should submit explicitly
%Add question about: creative changes to the project. 

This assignment is part of your final project. You will need to expand on the map class PROVIDED ON CANVAS. Take time to get familiar with the code and then begin modifying the class.

You are a player in the game tasked to explore the tiles on the map. Write a C++ program where you will generate a map that satisfies the minimum requirements specified in the project (see the Project 2 section of this workbook). You will need to have two lanes: one to represent when a player goes to Cub Training, and one to represent when a player goes Straight to the Pride Lands. 

Write code to enable a user to navigate through one lane on the map and visit the tiles. You should print out placeholder information for each type of tile, but you do not have to fully implement all of the tile types just yet. Make sure that your map is randomized for each new game, and that the generation rules are different for the two paths. For example, you could do something like this:

Straight to the Pridelands: %update to 52; player_index should get renamed
\begin{itemize}
    \item 20 Grasslands tiles; 29 special tiles; 1 start and 1 end tile. There are more special tiles in this path, because they do not have an advisor to put them on the clearer path.
    \item The player is more likely to go to a bad tile, since they do not have an advisor to help them avoid those tiles. They are more likely to make these mistakes early, before they learn. Therefore, for the first half of the lane there is a 25\% chance for a special tile to be a graveyard tile, and a 25\% chance for the special tile to be a hyena tile. In the second half of the lane, these odds are reduced to a 15\% chance for the graveyard, and a 15\% chance for the hyena tile.
    \item Since the players do not already have an advisor, they are more likely to seek one out. They have a 20\% chance of the random tile being an advisor tile.
    \item They are likely to learn from their mistakes and have an easier time finding an oasis as the game goes on. In the first half of the lane, there is a 5\% chance of a special tile being an oasis tile. For the second half, there is a 25\% chance of finding the oasis. 
    \item Challenge tiles are uniformly likely with a 25\% chance throughout the lane.
\end{itemize}

Cub Training:
\begin{itemize}
    \item 29 Grassland tiles; 20 special tiles; 1 start and 1 end tile.
    \item These players are less likely to visit a negative tile throughout the game, because their advisors will help them avoid them. So they have a 20\% chance for a graveyard and a 20\% for a hyena tile at any point during the game.
    \item These players already have an advisor and are less likely to seek one out, so there is a 15\% chance for a tile to be an advisor tile.
    \item These players are likely steered towards the easy path with an oasis at the beginning of their journey by their advisor. In the first half of the lane each special tile has a 25\% chance of being an oasis. In the second half, they have a 15\% chance of being an oasis.
    \item These players are more likely to want to test themselves after their training. In the first half of the lane there is a 20\% chance of finding a challenge tile. In the second half, there is a 30\% chance. 
\end{itemize}

This is only a suggestion -- you are more than welcome to design and balance the lane generation in any way you see fit. Your lanes must be random, and the two paths must have different generation rules, but beyond this is up to you. Read the Project 2 document carefully to ensure you meet all requirements before finalizing your game board class.

Any additional time you have can be dedicated to this week's homework, which is the Code Skeleton for your final project.

%Add more information about the submission requirements

\section{Homework: Code Skeleton}
%Explain what the code skeleton includes, and put a rubric here.

You will need to submit your code skeleton for your final project on Canvas for your homework this week. You should think about your code skeleton as an outline for how you will approach your project. You will need to determine how you are going to break down your code -- an outline for the major sections of your main function and what types of objects and functions will help you make each part.

For your code skeleton you will need:

\begin{itemize}
    \item For each class you create, your .h files should be complete with all the data members and member functions you will be using for each class. 
    \item For the class implementation .cpp files, you should fully implement simple functions like your getters, setters, and constructors. For more complex functions, you can include the function prototype with detailed comments.
    \item For your game driver, you should have an approximate outline of how you will approach your main function's code. Pseudocode-like comments are fine for this. An example of expected comments is available in the code skeleton assignment on Canvas.
\end{itemize}
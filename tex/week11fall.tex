\chapter*{Week 11: Objects Continued}
\addcontentsline{toc}{chapter}{Week 11: Objects Continued}
\setcounter{chapter}{12}
\setcounter{section}{0}

\begin{abstract}
This week you will:
\begin{enumerate}
    \item Learn the differences between header and source files
    \item Compile multiple files and organizing code across multiple files
    \item Learn how to nest objects
    \item Learn how to work with arrays of objects

\end{enumerate}
    
\end{abstract}

\section{Background}
\subsection{Header and Source Files}
Creating our own classes with various data members and functions increases the complexity of our program. Putting all of the code for our classes as well as the main functionality of our program into one .cpp file can become confusing for you as a programmer, and we need ways of reducing the visual clutter that this creates. This is why, as we increase the complexity of a program, we might need to create multiple files: header and source files, which capture the \textit{definition} and \textit{implementation} of the class, respectively.

\subsubsection{Header Files}

Header files have .h as their filename extensions. In a header file, we declare one or more of the complex structures (classes) we want to develop. In a class, we define member functions and member attributes. These functions and attributes are the building blocks of the class.

\begin{example}
    Example File 0: ClassName.h
    \begin{minted}{c++}
#include <iostream> 
using namespace std; 
class ClassName 
{ 
    public: 
    . 
    . 
    . 
    private: 
    . 
    . 
    . 
};

    \end{minted}
\end{example}

\begin{example}
    Example File 1: IntelligenceAgency.h
    \begin{minted}{c++}
#include <iostream> 
using namespace std; 
class IntelligenceAgency
{
    public:
        IntelligenceAgency();         // Default constructor
        IntelligenceAgency(string classifiedIntelligenceInput); // Parameterized const
        string organization_name;
        string getClassifiedIntelligence(){return classifiedIntelligence};
        void setClassifiedIntelligence(string classifiedIntelligenceInput)
        {classifiedIntelligence=classifiedIntelligenceInput};

    private:
        string classified_intelligence;
};
    \end{minted}
\end{example}

\subsubsection{Source File}

Source files are recognizable by the .cpp extension. In a source file, we implement the class defined in the header file. Since we are splitting the development of actual code for the class into a definition (header file) and an implementation (source file), we need to link the two somehow.

\begin{minted}{c++}
//IntelligenceAgency Example
#include "IntelligenceAgency.h"
\end{minted}

Or, more generally,

\begin{minted}{c++}
//General Example
#include "ClassName.h"
\end{minted}

In the source file, we will include the header file that defines the class so that the source file is “aware” of where we can retrieve the definition of the class. We must define the class definition in every source that wants to use our user defined data type (our class). When implementing each member function, our source files must tell the compiler that these functions are actually the methods defined in our class definition using the syntax that we showed earlier.

\subsubsection{How To Compile Multiple Files}

Once you create a project with one or more separate classes, it will be necessary to write multiple files (.h and .cpp) and test them before submitting them. You need to compile and execute your code using the command line. Make sure that you start by changing directories so that you are in the folder where your solution’s files are stored. In this example, our folder will be called \textbf{project2}. To change to this directory, use:

\mintinline{bash}{cd project2/}

When compiling from the command line, you need to specify every \mintinline{c++}{.cpp} file in your project. This means that when you call the g++ compiler, you need to explicitly name the files you’re compiling:

\mintinline{bash}{g++ -std=c++17 file1.cpp file2.cpp main.cpp}

The compiling command results in the creation of an executable file. Note that header files (.h) are NOT included in compilation commands. If you did not specify a name for this executable, it will be named \mintinline{c++}{a.out} by default. To execute this file, use the command:

\mintinline{bash}{./a.out}

You can add the \mintinline{bash}{-o} flag to your compiling command to give the output file a name:

\mintinline{bash}{g++ -o myName.out -std=c++17 file1.cpp file2.cpp main.cpp}

And then run it with:

\mintinline{bash}{./myName.out}

At this stage, you may also find it helpful to begin compiling with flags that tell you more information about possible errors in your code. These flags include:

\mintinline{bash}{-Wall}

Wall is short for "Warn All", which will turn on most of the warnings in C++. This will help identify various possible ways your code might go wrong, including array bounds errors and other helpful messages.

\mintinline{bash}{-Werror}

Werror will treat all warnings as errors. This will prevent you from skipping past the possible sources of error in your code, and you will need to make sure all warnings are resolved prior to compiling.

\mintinline{bash}{-Wpedantic}

This flag enables warnings that alert you about language constructs that are not ISO C or ISO C++ standard compliant. This is particularly helpful to identify constructs that may not be uniform in other compilers, which could cause problems with your code on other machines. This will help prevent instances where your code works on your personal computer, but does not work on CodeRunner or on the grader's computer. All together, your command line prompt will look something like this:

\mintinline{bash}{g++ -Wall -Werror -Wpedantic -o myName.out -std=c++17 file1.cpp file2.cpp main.cpp}

%maybe add a section on header guards? 
%add a section on arrays in objects?

\section{PreQuiz}
\begin{problem}
True or False: In C++, you must include a semicolon at the end of both struct and class definitions.
\end{problem}

\begin{problem}
True or False: In a class, public members are only accessible by member functions within the same class.
\end{problem}

\begin{problem}
True or False: Structs in C++ can only contain public data members, while classes can have both public and private data members.
\end{problem}

\begin{problem}
Fill in the blanks in the code below:
\begin{minted}{c++}
#include <iostream>
#include <string>
using namespace std;

struct ____ {  // Struct name
    string ____;  // Member for the name of the location
    int ____;     // Member for the population
};

int main() {
    ____ myLocation{"____", ____};  // Initialize with name and population

    cout << "Location: " << myLocation.____ << endl;
    cout << "Population: " << myLocation.____ << endl;

    return 0;
}
\end{minted}
    
\end{problem}

\begin{problem}
Fill in the blanks in the code below:
\begin{minted}{c++}
#include <iostream>
using namespace std;

class ____ {  // Class name
public:
    ____ (int a) {  // Constructor to initialize age
        ____ = a;
    }
    
    void ____ (int a) {  // Setter method for age
        age = ____;
    }
    
    int ____() const {  // Getter method for age
        return ____;
    }

private:
    int ____;  // Private data member to store age
};

int main() {
    ____ person(25);  // Create a Person object with age 25
    person.____(30);  // Set the person's age to 30
    cout << "The person's age is: " << person.____() << endl;  // Get and display the age
    return 0;
}
\end{minted}
\end{problem}

\section{Recitation}
\subsection{Spot the Error}
\begin{multipart}
The following code defines a class with a getter method to return a private member. Identify and fix the error.
\end{multipart}

\begin{minted}{c++}
#include <iostream>
using namespace std;

class MyClass {
public:
    MyClass(int val) { value = val; }

    int getValue() { return value; }

private:
    int value;
};

int main() {
    const MyClass obj(42);
    cout << obj.getValue() << endl;
    return 0;
}
\end{minted}

\begin{multipart}
The following code is supposed to define a class with a constructor to initialize a private data member. Identify and fix the error.
\end{multipart}

\begin{minted}{c++}
#include <iostream>
using namespace std;

class MyClass {
public:
    MyClass(int val) { valueX = val; }

    int getValue() const { return value; }

private:
    int value;
};

int main() {
    MyClass obj(10);
    cout << obj.getValue() << endl;
    return 0;
}
\end{minted}

\begin{multipart}
The following code is supposed to define a class with a private data member and a setter method. Identify and fix the error.
\end{multipart}

\begin{minted}{c++}
#include <iostream>
using namespace std;

class MyClass {
    int value;

    void setValue(int newValue) {
        value = newValue;
    }
};

int main() {
    MyClass obj;
    obj.setValue(20);
    return 0;
}
\end{minted}


\subsection{Coffee Shop Order System}
This exercise focuses on using structs and basic arrays in C++ to create a simple coffee shop ordering system. You will define structs to represent drinks and orders, as well as functions to manage and calculate the order's total.

To begin, define a struct called \texttt{Drink}. The \texttt{Drink} struct should have three fields: a \texttt{string} \texttt{name} for the drink’s name (e.g., "Latte" or "Espresso"), a \texttt{string} \texttt{size} for the drink’s size ("small," "medium," or "large"), and a \texttt{double} \texttt{price} for the drink’s cost. Next, create a function called \texttt{displayDrink} that takes a \texttt{Drink} object as an argument and outputs its details in a readable format such as:

\begin{quote}
\texttt{[Size] [Name]: \$[Price]}
\end{quote}

For example, a medium latte priced at \$3.50 should display as \texttt{Medium Latte: \$3.50}.

Now, create a second struct called \texttt{Order} to represent a customer’s order. The \texttt{Order} struct should contain an array of \texttt{Drink} objects and an integer \texttt{numDrinks} that tracks the number of drinks in the order. To keep things simple, assume a maximum of 10 drinks per order. Define \texttt{numDrinks} with an initial value of 0.

Next, write a function called \texttt{addDrink} that takes an \texttt{Order} and a \texttt{Drink} as arguments, adds the drink to the order’s \texttt{drinks} array, and increments \texttt{numDrinks}. If the order already has 10 drinks, output an error message and prevent additional drinks from being added.

Now create a function called \texttt{calculateTotal} that takes an \texttt{Order} as a parameter and returns the total price of all drinks in the order by summing the \texttt{price} of each \texttt{Drink} in the \texttt{drinks} array. Finally, implement a function called \texttt{displayOrder} that prints out each drink in the order using \texttt{displayDrink} and then displays the total order cost.

In the \texttt{main} function, test your structs and functions by creating a few \texttt{Drink} objects, an \texttt{Order} object, and adding the drinks to the order. Finally, call \texttt{displayOrder} to print the details of the order and verify that the total is calculated correctly.

\begin{multipart}
    Write out the steps you would use to solve this problem by hand as pseudocode.
\end{multipart}

\vspace{8cm}

\begin{multipart}
    Write three possible lines you can include in your file to later test your program. Try to pick values that will test different aspects of your program. Follow the steps you wrote for these values to find your result, and verify it.
\end{multipart}

\vspace{5cm}

\begin{multipart}
    Implement your solution in C++ using VS Code. Save, compile, and run it. Test the program with different input files, including boundary conditions.
\end{multipart}

\vspace{5cm}

\section{Homework 8}
%all object content is fair game here. Make sure to have 1 question with an array of objects

\textbf{Warning: You are not allowed to use global variables for this assignment.}

All function names, return types, and parameters must precisely match those shown. You may not use pass by reference or otherwise modify the function prototypes. You are welcome to create additional functions that may help streamline your code. 

\textbf{Note:} Questions 1 through 3 are grouped together as Part I, and Questions 4 through 6 are grouped together as Part II.

In Part I, you'll develop a basic Library System by creating a Library class to manage a collection of books. In Part II, you'll develop a Quest Style Game. 

\subsection{Library Class} 


Create a class Library by splitting the code into the following files:
\begin{itemize}
    \item A header file \mintinline{c++}{Library.h} to declare the definition of the class
    \item  An implementation file \mintinline{c++}{Library.cpp} to implement the class defined in the header
    \item  A driver file \mintinline{c++}{LibraryDriver.cpp} that contains the main function
\end{itemize}

\textbf{For Question 1, develop and validate your solution in VS Code. Once you are happy with your solution, go to coderunner and paste \mintinline{c++}{Library.h} and \mintinline{c++}{Library.cpp} files into the answer box. (Note: You are not expected to complete the implementation of \mintinline{c++}{getAveragePageCount()} function for Q1, but do include the function declaration within the header file).} 

\vspace{10pt}

The Library class comprises the following attributes:

\vspace{10pt}

\textbf{Data members (private)}

\renewcommand{\arraystretch}{1.5}
\begin{longtable}{|p{1.5in}|p{1.5in}|p{2.9in}|}
\hline
\textbf{Member Type} & \textbf{Member Name} & \textbf{Description} \\ \hline

\mintinline{c++}{string} & \mintinline{c++}{_name} & The name of the Library \\ \hline

\mintinline{c++}{const static int} & \mintinline{c++}{_MAX_BOOKS} & The maximum number of books the library can hold; will be 10 for this question \\ \hline

\mintinline{c++}{int[]} & \mintinline{c++}{_page_counts} & Array containing the page counts of books in the library. The size of this array is \mintinline{c++}{_MAX_BOOKS} \\ \hline

\mintinline{c++}{int} & \mintinline{c++}{_current_books} & The current number of books in the Library \\ \hline

\end{longtable}

\textbf{Member Functions (public)}

\renewcommand{\arraystretch}{1.5} 
\begin{longtable}{|p{2.0in}|p{4.0in}|}
\hline
\textbf{Function} & \textbf{Description} \\ \hline

\texttt{Default constructor} & Creates a new instance of \mintinline{c++}{Library} by setting \mintinline{c++}{_name} to an empty string, \mintinline{c++}{_current_books} to 0, and each entry in \mintinline{c++}{_page_counts} to 0. \\ \hline

\mintinline{c++}{Library(string)} & Creates a new instance of \mintinline{c++}{Library} with \mintinline{c++}{_name} as the string parameter. The \mintinline{c++}{_current_books} is set to 0, and each entry in the \mintinline{c++}{_page_counts} array is set to 0. \\ \hline

\mintinline{c++}{Library(string, int[], int)} & Creates a new instance of \mintinline{c++}{Library} with \mintinline{c++}{_name} as the string parameter, \mintinline{c++}{_current_books} as the integer parameter, and fills the \mintinline{c++}{_page_counts} array. See Function Specification table below for more details. \\ \hline

\mintinline{c++}{string getName()} & Returns the \mintinline{c++}{_name} of the \mintinline{c++}{Library}. \\ \hline

\mintinline{c++}{int getCurrentBooks()} & Returns the \mintinline{c++}{_current_books} of the \mintinline{c++}{Library}. \\ \hline

\mintinline{c++}{void setName(string)} & Sets the \mintinline{c++}{_name} to the value of the provided string parameter. \\ \hline

\mintinline{c++}{int getPageCount(int)} & Returns the page count at a specified index in \mintinline{c++}{_page_counts}, or -1 if the index is out of bounds. \\ \hline

\mintinline{c++}{bool addPageCount(int)} & Returns \mintinline{c++}{true} if the new page count can be added to the \mintinline{c++}{_page_counts} array. If \mintinline{c++}{_current_books} is already equal to \mintinline{c++}{_MAX_BOOKS}, returns \mintinline{c++}{false}. See Function Specification table below for more details. \\ \hline

\mintinline{c++}{double getAveragePageCount()} & Calculates and returns the average page count of the books in the \mintinline{c++}{Library}. See Question 2 for more details. \\ \hline

\end{longtable}


\textbf{Function Specifications:}

\renewcommand{\arraystretch}{1.5}
\begin{longtable}{|p{1.7in}|p{4.3in}|}
\hline
\textbf{Function:} \mintinline{c++}{Library(string, int[], int)} & \mintinline{c++}{Library(string name, int page_counts[], int arr_size)} \\ \hline

\textbf{Purpose:} & This parameterized constructor creates a new instance of the \mintinline{c++}{Library} class. \newline
- Sets \mintinline{c++}{_name} to \mintinline{c++}{name}. \newline
- Sets \mintinline{c++}{_current_books} to \mintinline{c++}{arr_size} if \mintinline{c++}{arr_size} doesn't exceed \mintinline{c++}{_MAX_BOOKS}; otherwise, \mintinline{c++}{_current_books} is set to \mintinline{c++}{_MAX_BOOKS}. \newline
- Copies elements from \mintinline{c++}{page_counts[]} into the \mintinline{c++}{_page_counts[]} array, up to the size indicated by \mintinline{c++}{_current_books}. \\ \hline

\textbf{Parameters:} & 
\mintinline{c++}{string} \mintinline{c++}{name} - The name of the Library. \newline
\mintinline{c++}{int} \mintinline{c++}{page_counts[]} - Array containing page counts of books in the library. \newline
\mintinline{c++}{int} \mintinline{c++}{arr_size} - The size of the \mintinline{c++}{page_counts[]} array. \\ \hline

\textbf{Return Value:} & N/A \\ \hline

\textbf{Error Handling:} & 
If \mintinline{c++}{arr_size} exceeds \mintinline{c++}{_MAX_BOOKS}, only the first \mintinline{c++}{_MAX_BOOKS} elements are stored in the \mintinline{c++}{_page_counts[]} array. Do NOT print any other statements to handle this scenario. \\ \hline

\textbf{Example:} & 


\begin{example}


\begin{minted}[breaklines=true]{c++}

// Assume the proper libraries are included
// Assume the proper implementation of the class is included

int main() {
    string name = "City Library";
    int arr_size = 5;
    int page_counts[arr_size] = {250, 300, 150, 500, 400};
    Library new_library = Library(name, page_counts, arr_size);
}
\end{minted}
\end{example}

\textbf{Expected Contents of \mintinline{c++}{new_library} Object:}

\vspace{5pt}
\mintinline{c++}{_name} = ``City Library" \newline
\mintinline{c++}{_MAX_BOOKS} = 10 \newline
\mintinline{c++}{_current_books} = 5 \newline
\mintinline{c++}{_page_counts[]} = \{250, 300, 150, 500, 400, 0, 0, 0, 0, 0\}
\vspace{5pt}
\\ \hline
\end{longtable}


\renewcommand{\arraystretch}{1.5}
\begin{longtable}{|p{1.7in}|p{4.3in}|}
\hline
\textbf{Function:} \mintinline{c++}{addPageCount(int)} & \mintinline{c++}{bool addPageCount(int pages)} \\ \hline

\textbf{Purpose:} & Adds the provided page count to the \mintinline{c++}{_page_counts} array. \\ \hline

\textbf{Parameters:} & 
\mintinline{c++}{int} \mintinline{c++}{pages} - The number of pages to be added. \\ \hline

\textbf{Return Value:} & \mintinline{c++}{bool} : Returns a boolean indicating whether the new page count was successfully added to the \mintinline{c++}{_page_counts} array. \\ \hline

\textbf{Error Handling:} & 
- If \mintinline{c++}{_current_books} is already equal to \mintinline{c++}{_MAX_BOOKS}, do not add or modify any contents, and return \mintinline{c++}{false}. \newline
- If the \mintinline{c++}{pages} count is non-positive, do not add or modify any contents, and return \mintinline{c++}{false}. \newline
- Return \mintinline{c++}{true} if the new entry is successfully added. \\ \hline

\textbf{Example:} & 

\begin{example}

\begin{minted}[breaklines=true]{c++}

// Assume the proper libraries are included
// Assume the proper implementation of the Library class is included

int main() {
    string name = "City Library";
    int arr_size = 5;
    int page_counts[arr_size] = {250, 300, 150, 500, 400};
    Library new_library = Library(name, page_counts, arr_size);
    cout << new_library.addPageCount(90) << endl;
}
\end{minted}
\end{example}

\textbf{Expected Return Value:} 

\vspace{5pt}
\texttt{true} 
\vspace{5pt}

\textbf{Expected Contents of \mintinline{c++}{new_library} Object after Function Call:} 

\vspace{5pt}
\mintinline{c++}{_name} = ``City Library" \newline
\mintinline{c++}{_MAX_BOOKS} = 10 \newline
\mintinline{c++}{_current_books} = 6 \newline
\mintinline{c++}{_page_counts[]} = \{250, 300, 150, 500, 400, 90, 0, 0, 0, 0\}
\vspace{5pt} \\ \hline

\end{longtable}
\newpage
\subsection{Find Average Page Count of Books in a Library}

Write a member function \mintinline{c++}{getAveragePageCount()} in the Library class to find the average page count of the books in \mintinline{c++}{_pageCounts}.

\textbf{For Question 2, develop and validate your solution in VS Code. Once you are happy with your solution, go to coderunner on Canvas paste code from \mintinline{c++}{Library.h} and \mintinline{c++}{Library.cpp} into the answer box.} 

\renewcommand{\arraystretch}{1.5}
\begin{longtable}{|p{1.7in}|p{4.3in}|}
\hline
\textbf{Function:} \mintinline{c++}{getAveragePageCount()} & \mintinline{c++}{double getAveragePageCount()} \\ \hline

\textbf{Purpose:} & Calculates and returns the average page count of the books in the \mintinline{c++}{Library}. \\ \hline

\textbf{Parameters:} & 
This member function takes no parameters. \\ \hline

\textbf{Return Value:} & \mintinline{c++}{double} : The average page count of all books in the library.

The function should not print anything. \\ \hline

\textbf{Error Handling:} & 
Return \mintinline{c++}{0} if the \mintinline{c++}{_current_books} is zero. \\ \hline

\textbf{Example:} & 

\begin{example}

Note: This is only an example usage of the function; you need to develop your own main function to fulfill the requirement for this problem.

\begin{minted}[breaklines=true]{c++}

// Assume the proper libraries are included
// Assume the proper implementation of the library class is included

int main()
{
    string name = "City Library";
    int arr_size = 5;
    int page_counts[arr_size] = {250, 300, 150, 500, 400};
    Library new_library = Library(name, page_counts, arr_size);
    cout << fixed<< setprecision(2) << new_library.getAveragePageCount() << endl;
}


\end{minted}
\end{example}


\textbf{Expected Output:} 

\vspace{5pt}
The expected output after the function call:

320.00

\\ \hline

\end{longtable}


\newpage
\subsection{Library with the Largest Average Page Count}

Write a function \mintinline{c++}{findLibraryWithLargestAveragePageCount()} to find the \mintinline{c++}{Library} in the array with the largest average page count.

\vspace{5pt}

\textbf{For Question 3, develop and validate your solution in VS Code. Once you are happy with your solution, go to coderunner on Canvas and paste code from \mintinline{c++}{Library.h}, \mintinline{c++}{Library.cpp}, and the function \mintinline{c++}{findLibraryWithLargestAveragePageCount} into the answer box.} 

\renewcommand{\arraystretch}{1.5}
\begin{longtable}{|p{1.7in}|p{4.3in}|}
\hline
\textbf{Function:} \mintinline{c++}{findLibraryWithLargest}\newline  \mintinline{c++}{AveragePageCount(Library[],}\newline \mintinline{c++}{const int)} & \mintinline{c++}{string findLibraryWithLargestAveragePageCount(Library libraries[],} \newline \mintinline{c++}{const int ARR_SIZE)} \\ \hline

\textbf{Purpose:} & Finds the name of the library in the \mintinline{c++}{libraries} array with the largest average page count of the libraries. \\ \hline

\textbf{Parameters:} & 
\mintinline{c++}{Library} \mintinline{c++}{libraries[]} - Array of Library objects. \newline
\mintinline{c++}{const int} \mintinline{c++}{ARR_SIZE} - The size of the \mintinline{c++}{libraries} array. \\ \hline

\textbf{Return Value:} & \mintinline{c++}{string} : The name of the library with the largest average page count. The function should not print anything. \\ \hline

\textbf{Error Handling:} & 
- If multiple libraries have the same largest average page count, return the name of the first library with the largest average page count found in the array. \newline
- All libraries[] array will be non-empty. \\ \hline

\textbf{Example:} & 

\begin{example}

\begin{minted}[breaklines=true]{c++}

// Assume the proper libraries are included
// Assume the proper implementation of the Library class is included

int main() {
    const int ARR_SIZE = 2;
    int page_counts1[3] = {250, 300, 350};
    int page_counts2[4] = {200, 300, 250, 400};
    Library library1 = Library("Downtown Library", page_counts1, 3);
    Library library2 = Library("Uptown Library", page_counts2, 4);
    Library libraries[ARR_SIZE] = {library1, library2};
    cout << findLibraryWithLargestAveragePageCount(libraries, ARR_SIZE) << endl;
}

\end{minted}
\end{example}


\textbf{Expected Output:} 

\vspace{5pt}
\texttt{Downtown Library} 
\vspace{5pt}

\textbf{Explanation:} 

\vspace{5pt}
Average page count of \mintinline{c++}{library1} = 300.00 \newline
Average page count of \mintinline{c++}{library2} = 287.50 \newline

\mintinline{c++}{library1} has the largest average page count, so we return the name of \mintinline{c++}{library1}, i.e., \mintinline{c++}{Downtown Library}.
\vspace{5pt} \\ \hline

\end{longtable}


\vspace{5pt}

\subsection{Character Class} 

Develop a class called Character that will represent either a player or a creature. This class will be split into:

\begin{itemize}
    \item A header file \mintinline{c++}{Character.h} to declare the definition of the class
    \item  An implementation file \mintinline{c++}{Character.cpp} to implement the class defined in the header
    \item  A driver file \mintinline{c++}{CharacterDriver.cpp} that contains the main function
\end{itemize}


\textbf{For Question 4, develop and validate your solution in VS Code. Once you are happy with your solution, go to coderunner on Canvas and paste \mintinline{c++}{Character.h} and \mintinline{c++}{Character.cpp} into the answer box.} 

\vspace{10pt}

\textbf{Data members (private)}

\renewcommand{\arraystretch}{1.5}
\begin{longtable}{|p{1.5in}|p{1.5in}|p{2.9in}|}
\hline
\textbf{Member Type} & \textbf{Member Name} & \textbf{Description} \\ \hline

\mintinline{c++}{string} & \mintinline{c++}{_name} & The characters name \\ \hline

\mintinline{c++}{double} & \mintinline{c++}{_health} & The characters health points \\ \hline

\mintinline{c++}{int} & \mintinline{c++}{_mana} & The amount of magical energy that the character has \mintinline{c++}{_mana} \\ \hline

\mintinline{c++}{char} & \mintinline{c++}{_status} & Current status: `A' for Active, `C' for Cursed, `F' for Frozen\\ \hline

\mintinline{c++}{bool} & \mintinline{c++}{_isCreature} & Specifies if the character is a mystical creature\\ \hline

\end{longtable}


\textbf{Member Functions (public)}

\renewcommand{\arraystretch}{1.5} 
\begin{longtable}{|p{2.0in}|p{4.0in}|}
\hline
\textbf{Function} & \textbf{Description} \\ \hline

\texttt{Default constructor} & Initializes \mintinline{c++}{_mana} and \mintinline{c++}{_health} to 0, \mintinline{c++}{_name} to an empty string, \mintinline{c++}{_status} to `A', and \mintinline{c++}{_isCreature} to false \\ \hline

\mintinline{c++}{Character(string, double, int,} \newline \mintinline{c++}{char, bool)} & Initializes the data members with the provided values.
See Function Specification table below for more details \\ \hline

\mintinline{c++}{string getName()} & Returns the \mintinline{c++}{_name} of the \mintinline{c++}{character}. \\ \hline

\mintinline{c++}{double getHealth()} & Returns the \mintinline{c++}{_health} of the character. \\ \hline

\mintinline{c++}{int getMana()} & Returns the \mintinline{c++}{_mana} of the character. \\ \hline

\mintinline{c++}{char getStatus()} & Returns the current status of the character. \\ \hline

\mintinline{c++}{bool getIsCreature()} & Returns if the character is a creature\\ \hline

\mintinline{c++}{void setName(string name)} & Sets the \mintinline{c++}{_name} to the value of the provided string parameter. \\ \hline

\mintinline{c++}{void setHealth(double health)} & Sets the amount of \mintinline{c++}{_health} the character possesses to the given value health only if it is a non-negative value, else it is not changed \\ \hline

\mintinline{c++}{void setStatus(char status)} & Sets the status to one of the allowed characters (A, C, or F) to the given value status only if it is one among A, C or F, else it is not changed \\ \hline

\mintinline{c++}{void setMana(int mana)} & Sets the amount of \mintinline{c++}{_mana} the character possesses to the given value mana only if it is a non-negative value, else it is not changed \\ \hline

\mintinline{c++}{void setIsCreature(bool} \newline \mintinline{c++}{is_enemy)} & Sets if the character is a creature \\ \hline

\mintinline{c++}{void printStats()} & Prints out the character's stats
See Function Specification table below for more details
 \\ \hline

\end{longtable}

\textbf{Function Specifications:}

\renewcommand{\arraystretch}{1.5}
\begin{longtable}{|p{1.7in}|p{4.3in}|}
\hline
\textbf{Function:} \mintinline{c++}{Character(string, double,} \newline \mintinline{c++}{int, char, bool)} & \mintinline{c++}{Character(string name, double health, int mana, char status,} \newline \mintinline{c++}{bool isCreature)} \\ \hline

\textbf{Purpose:} & This parameterized constructor creates a new instance of the \mintinline{c++}{Character} class and sets the data members as provided. \\ \hline

\textbf{Parameters:} & 
\mintinline{c++}{string name} - The name of the character. \newline
\mintinline{c++}{double health} - Health points of the character. \newline
\mintinline{c++}{int mana} - Amount of magical energy the character has. \newline
\mintinline{c++}{char status} - The current status of the character (`A' for Active, `C' for Cursed, `F' for Frozen). \newline
\mintinline{c++}{bool isCreature} - Specifies if the character is a mystical creature. \\ \hline

\textbf{Return Value:} & N/A \\ \hline

\textbf{Error Handling:} & 
\mintinline{c++}{_health} is set to the value of \mintinline{c++}{health} only if it is non-negative; otherwise, it is set to 0. \newline
\mintinline{c++}{_mana} is set to \mintinline{c++}{mana} only if it is non-negative; otherwise, it is set to 0. \newline
\mintinline{c++}{_status} is set to \mintinline{c++}{status} if it is `A', `C', or `F'; otherwise, it is set to `A'. \\ \hline

\textbf{Example:} & 

\begin{example}

Note: This is only an example usage of the function; you need to develop your own main function to fulfill the requirement for this problem.

\begin{minted}[breaklines=true]{c++}

// Assume the proper libraries are included
// Assume the proper implementation of the Character class is included

int main() {
    Character character1("Elara", 8.2, 12, 'F', true);
}
\end{minted}
\end{example}

\textbf{Expected Contents of \mintinline{c++}{character1} Object:} 

\vspace{5pt}
\mintinline{c++}{_name = "Elara"} \newline
\mintinline{c++}{_health = 8.2} \newline
\mintinline{c++}{_mana = 12} \newline
\mintinline{c++}{_status = 'F'} \newline
\mintinline{c++}{_isCreature = true}
\vspace{5pt} \\ \hline
\end{longtable}


\renewcommand{\arraystretch}{1.5}
\begin{longtable}{|p{1.7in}|p{4.3in}|}
\hline
\textbf{Function:} \mintinline{c++}{printStats()} & \mintinline{c++}{void printStats()} \\ \hline

\textbf{Purpose:} & This member function prints the stats of the character. All \mintinline{c++}{double} values are printed up to 2 decimal places. \\ \hline

\textbf{Parameters:} & 
This function takes no parameters. \\ \hline

\textbf{Return Value:} & 
\mintinline{c++}{void} : The function does not return any value. \\ \hline

\textbf{Example:} & 

\begin{example}

\begin{minted}[breaklines=true]{c++}

// Assume the proper libraries are included
// Assume the proper implementation of the Character class is included

int main() {
    Character character1("Elara", 8.2, 12, 'F', true);
    character1.printStats();
}
\end{minted}
\end{example}

\textbf{Expected Output:} 

\vspace{5pt}
Elara's stats: \newline
Health: 8.20 \newline
Status: F \newline
Mana: 12 \newline
Is Creature: Yes
\vspace{5pt} \\ \hline
\end{longtable}

%%%%%%%%%%%%%%%%%%%%%%%%%%%Q5%%%%%%%%%%%%%%%%%%%%%%%%%%%%
\newpage
\subsection{Game Class} 

Create a class Game by splitting the code into the following files:

\begin{itemize}
    \item A header file \mintinline{c++}{Game.h} to declare the definition of the class
    \item  An implementation file \mintinline{c++}{Game.cpp} to implement the class defined in the header
    \item  A driver file \mintinline{c++}{GameDriver.cpp} that contains the main function to test the class
\end{itemize}

\textbf{For Question 5, develop and validate your solution in VS Code. Once you are happy with your solution, go to coderunner on Canvas and paste \mintinline{c++}{Game.h}, \mintinline{c++}{Game.cpp}, \mintinline{c++}{Character.h}, and \mintinline{c++}{Character.cpp} into the answer box.} 

\vspace{10pt}

\textbf{Data members (private)}

\renewcommand{\arraystretch}{1.5}
\begin{longtable}{|p{1.5in}|p{1.5in}|p{2.9in}|}
\hline
\textbf{Member Type} & \textbf{Member Name} & \textbf{Description} \\ \hline

\mintinline{c++}{Character} & \mintinline{c++}{_players[2]} & Array storing player objects \\ \hline

\mintinline{c++}{Character} & \mintinline{c++}{_creatures[2]} & Array storing mystical creature characters \\ \hline

\mintinline{c++}{int} & \mintinline{c++}{_num_players} & Current number of players in the game \\ \hline

\mintinline{c++}{int} & \mintinline{c++}{_num_creatures} & Current number of creatures in the game \\ \hline

\end{longtable}


\textbf{Member Functions (public)}

\renewcommand{\arraystretch}{1.5} 
\begin{longtable}{|p{2.0in}|p{4.0in}|}
\hline
\textbf{Function} & \textbf{Description} \\ \hline

\texttt{Default constructor} & Creates a new instance of \texttt{Game} with empty \mintinline{c++}{_players} and \mintinline{c++}{_creatures} arrays and sets \mintinline{c++}{_num_players} and \mintinline{c++}{_num_creatures} to 0. \\ \hline

\mintinline{c++}{Game(Character[], Character[],} \newline \mintinline{c++}{int, int)} & Creates a new instance of \texttt{Game} with the provided parameters. See Function Specification table below for more details. \\ \hline

\mintinline{c++}{int getNumPlayers()} & Returns the current number of players \mintinline{c++}{_num_players}. \\ \hline

\mintinline{c++}{int getNumCreatures()} & Returns the current number of creatures \mintinline{c++}{_num_creatures}. \\ \hline

\mintinline{c++}{void setPlayersList} \newline \mintinline{c++}{(Character[], int)} & Sets the \mintinline{c++}{_players} array with the provided array of \texttt{Character} objects. If the number of objects exceeds 2, only the first two objects are considered. Updates \mintinline{c++}{_num_players} accordingly. \\ \hline

\mintinline{c++}{void setCreaturesList} \newline \mintinline{c++}{(Character[], int)} & Sets the \mintinline{c++}{_creatures} array with the provided array of \texttt{Character} objects. If the number of objects exceeds 2, only the first two objects are considered. Updates \mintinline{c++}{_num_creatures} accordingly. \\ \hline

\mintinline{c++}{bool setPlayer(int, Character)} & Replaces a player object at the specified index in the \mintinline{c++}{_players} array. See Function Specification table below for more details. \\ \hline

\mintinline{c++}{Character getPlayer(string)} & Returns an object from the \mintinline{c++}{_players} array based on the provided name. If no object matches the name, returns a new blank \texttt{Character} object. \\ \hline

\mintinline{c++}{bool setCreature} \newline \mintinline{c++}{(int, Character)} & Replaces a creature object at the specified index in the \mintinline{c++}{_creatures} array. See Function Specification table below for more details. \\ \hline

\mintinline{c++}{Character getCreature(string)} & Returns an object from the \mintinline{c++}{_creatures} array based on the provided name. If no object matches the name, returns a new blank \texttt{Character} object. \\ \hline

\mintinline{c++}{int findPlayer(string)} & Returns the index of the player object in the \mintinline{c++}{_players} array based on the provided name. See Function Specification table below for more details. \\ \hline

\mintinline{c++}{int findCreature(string)} & Returns the index of the creature object in the \mintinline{c++}{_creatures} array based on the provided name. See Function Specification table below for more details. \\ \hline

\mintinline{c++}{void printAllStats()} & Prints stats of all the players and creatures. See Function Specification table below for more details. \\ \hline

\end{longtable}












\textbf{Function Specifications:}

\renewcommand{\arraystretch}{1.5}
\begin{longtable}{|p{1.7in}|p{4.3in}|}
\hline
\textbf{Function:} \mintinline{c++}{Game(Character[],} \newline \mintinline{c++}{Character[], int, int)} & \mintinline{c++}{Game(Character players[], Character creatures[], int num_players,} \mintinline{c++}{int num_creatures)} \\ \hline

\textbf{Purpose:} & This parameterized constructor creates a new instance of the \mintinline{c++}{Game} class. \newline
- Sets \mintinline{c++}{_num_players} to \mintinline{c++}{num_players}. \newline
- Sets \mintinline{c++}{_num_creatures} to \mintinline{c++}{num_creatures}. \newline
- Assigns elements from the \mintinline{c++}{players[]} array to the \mintinline{c++}{_players[]} array. \newline
- Assigns elements from the \mintinline{c++}{creatures[]} array to the \mintinline{c++}{_creatures[]} array. \\ \hline

\textbf{Parameters:} & 
\mintinline{c++}{Character players[]} - Array of player objects. \newline
\mintinline{c++}{Character creatures[]} - Array of creature objects. \newline
\mintinline{c++}{int num_players} - Number of players in the array. \newline
\mintinline{c++}{int num_creatures} - Number of creatures in the array. \\ \hline

\textbf{Return Value:} & N/A \\ \hline

\textbf{Error Handling:} & 
If the size of either the \mintinline{c++}{players[]} or \mintinline{c++}{creatures[]} array is greater than 2, only the first two elements are stored in the respective arrays. \\ \hline

\textbf{Example:} & 

\begin{example}

Note: This is only an example usage; you need to develop your own function to fulfill the requirement.

\begin{minted}[breaklines=true]{c++}

// Assume the proper libraries are included
// Assume the proper implementation of the Character and Game classes is included

int main() {
    Character player1("Elise", 100.0, 50, 'A', false);
    Character player2("Pawin", 80.0, 30, 'A', false);
    Character creature1("Goblin", 60.0, 20, 'C', true);
    Character creature2("Dragon", 150.0, 100, 'F', true);

    Character players[2] = {player1, player2};
    Character creatures[2] = {creature1, creature2};

    Game new_game(players, creatures, 2, 2);
}
\end{minted}
\end{example}

\textbf{Expected Contents of \mintinline{c++}{new_game} Object:} 

\vspace{5pt}
\mintinline{c++}{_num_players = 2} \newline
\mintinline{c++}{_num_creatures = 2} \newline
\mintinline{c++}{_players[]} = \{ ``Elise", ``Pawin" \} \newline
\mintinline{c++}{_creatures[]} = \{ ``Goblin", ``Dragon" \} \newline 

Note: For demonstration purposes, we are only displaying the names of the \mintinline{c++}{_players[]} and \mintinline{c++}{_creatures[]}. 
\vspace{5pt} \\ \hline
\end{longtable}


\renewcommand{\arraystretch}{1.5}
\begin{longtable}{|p{1.7in}|p{4.3in}|}
\hline
\textbf{Function:} \mintinline{c++}{setPlayer(int, Character)} & \mintinline{c++}{bool setPlayer(int index, Character new_player)} \\ \hline

\textbf{Purpose:} & Replaces the player object in the \mintinline{c++}{_players} array at the specified index with the new \mintinline{c++}{Character} object. \\ \hline

\textbf{Parameters:} & 
\mintinline{c++}{int index} - Index in the \mintinline{c++}{_players} array. \newline
\mintinline{c++}{Character new_player} - New \mintinline{c++}{Character} object to replace the existing one. \\ \hline

\textbf{Return Value:} & \mintinline{c++}{bool} : Returns \mintinline{c++}{true} if the replacement was successful; otherwise, returns \mintinline{c++}{false} if the index is out of bounds. \\ \hline

\textbf{Example:} & 

\begin{example}

\begin{minted}[breaklines=true]{c++}

int main() {
    Character player1("Elise", 100.0, 50, 'A', false);
    Character player2("Pawin", 80.0, 30, 'A', false);
    Character new_player("Clara", 120.0, 40, 'A', false);
    Character players[2] = {player1, player2};
    Character creatures[1] = {Character("Goblin", 60.0, 20, 'C', true)};
    Game game(players, creatures, 2, 1);
    int index = game.findPlayer("Elise");
    cout << "Status of setting player at index " << index << ": " << game.setPlayer(index, new_player) << endl;
}
\end{minted}
\end{example}

\textbf{Expected Output:} 

\vspace{5pt}
\mintinline{c++}{Status of setting player at index 0: 1}
\vspace{5pt} \\ \hline
\end{longtable}


\renewcommand{\arraystretch}{1.5}
\begin{longtable}{|p{2.0in}|p{4.0in}|}
\hline
\textbf{Function:} \mintinline{c++}{setCreature(int, Character)} & \mintinline{c++}{bool setCreature(int index, Character new_creature)} \\ \hline

\textbf{Purpose:} & Replace the creature object in the \mintinline{c++}{_creatures} array at the specified index with the new \texttt{Character} object. \\ \hline

\textbf{Parameters:} & 
\mintinline{c++}{int index} - Index of the creature object in the \mintinline{c++}{_creatures} array. \newline
\mintinline{c++}{Character} \mintinline{c++}{new_creature} - The new \mintinline{c++}{Character} object which replaces an older object in the \mintinline{c++}{_creatures} array. \\ \hline

\textbf{Return Value:} & \mintinline{c++}{bool} : Returns \mintinline{c++}{true} if the new object replaces an older object in the \mintinline{c++}{_creatures} array; returns \mintinline{c++}{false} if the index is out of bounds. \\ \hline

\textbf{Error Handling:} & 
Returns \mintinline{c++}{false} if the attribute index is not within the size of the \mintinline{c++}{_creatures} array. \\ \hline

\textbf{Example:} & 

\begin{example}
\begin{minted}[breaklines=true]{c++}

int main() {
    Character player1("Brian", 100, 50, 'A', false);
    Character player2("Sam", 80, 100, 'A', false);
    Character creature1("Hydra", 200, 60, 'C', true);
    Character creature2("Minotaur", 150, 30, 'C', true);
    Character players[2] = {player1, player2};
    Character creatures[2] = {creature1, creature2};
    int num_players = 2;
    int num_creatures = 2;
    Game new_game = Game(players, creatures, num_players, num_creatures);

    int index = new_game.findCreature("Minotaur");
    cout << "Status of setting creature at index " << index << " is " << new_game.setCreature(index, creature1) << endl;

    index = new_game.findCreature("Phoenix");
    cout << "Status of setting creature at index " << index << " is " << new_game.setCreature(index, creature2) << endl;
}
\end{minted}
\end{example}

\textbf{Expected Output:} 

\vspace{5pt}
Status of setting creature at index 1 is 1 \newline
Status of setting creature at index -1 is 0
\vspace{5pt}
\\ \hline
\end{longtable}

\newpage
\renewcommand{\arraystretch}{1.5}
\begin{longtable}{|p{2.0in}|p{4.0in}|}
\hline
\textbf{Function:} \mintinline{c++}{findPlayer(string)} & \mintinline{c++}{int findPlayer(string name)} \\ \hline

\textbf{Purpose:} & Finds the index of the player object in the \mintinline{c++}{_players} array whose name matches the provided name. \\ \hline

\textbf{Parameters:} & 
\textbf{string} \texttt{name} - Name of the player \\ \hline

\textbf{Return Value:} & \mintinline{c++}{int} : Returns the index of the object in the \mintinline{c++}{_players} array. Returns -1 if the name doesn't match any object. \\ \hline

\textbf{Example:} & 

\begin{example}
\begin{minted}[breaklines=true]{c++}

int main() {
    Character player1("Brian", 100, 50, 'A', false);
    Character player2("Sam", 80, 100, 'A', false);
    Character players[2] = {player1, player2};
    int num_players = 2;
    int num_creatures = 0;
    Game new_game = Game(players, nullptr, num_players, num_creatures);

    cout << "Status of finding player with name Brian: " << new_game.findPlayer("Brian") << endl;
    cout << "Status of finding player with name Jenna: " << new_game.findPlayer("Jenna") << endl;
}
\end{minted}
\end{example}

\textbf{Expected Output:} 

\vspace{5pt}
Status of finding player with name Brian: 0 \newline
Status of finding player with name Jenna: -1
\vspace{5pt}
\\ \hline
\end{longtable}

\newpage
\renewcommand{\arraystretch}{1.5}
\begin{longtable}{|p{2.0in}|p{4.0in}|}
\hline
\textbf{Function:} \mintinline{c++}{findCreature(string)} & \mintinline{c++}{int findCreature(string name)} \\ \hline

\textbf{Purpose:} & Finds the index of the creature object in the \mintinline{c++}{_creatures} array whose name matches the provided name. \\ \hline

\textbf{Parameters:} & 
\textbf{string} \texttt{name} - Name of the creature \\ \hline

\textbf{Return Value:} & \mintinline{c++}{int} : Returns the index of the object in the \mintinline{c++}{_creatures} array. Returns -1 if the name doesn't match any object. \\ \hline

\textbf{Example:} & 

\begin{example}
\begin{minted}[breaklines=true]{c++}

int main() {
    Character creature1("Hydra", 200, 60, 'C', true);
    Character creature2("Minotaur", 150, 30, 'C', true);
    Character creatures[2] = {creature1, creature2};
    int num_players = 0;
    int num_creatures = 2;
    Game new_game = Game(nullptr, creatures, num_players, num_creatures);

    cout << "Status of finding creature with name Hydra: " << new_game.findCreature("Hydra") << endl;
    cout << "Status of finding creature with name Phoenix: " << new_game.findCreature("Phoenix") << endl;
}
\end{minted}
\end{example}

\textbf{Expected Output:} 

\vspace{5pt}
Status of finding creature with name Hydra: 0 \newline
Status of finding creature with name Phoenix: -1
\vspace{5pt}
\\ \hline
\end{longtable}



















\renewcommand{\arraystretch}{1.5}
\begin{longtable}{|p{1.7in}|p{4.3in}|}
\hline
\textbf{Function:} \mintinline{c++}{printAllStats()} & \mintinline{c++}{void printAllStats()} \\ \hline

\textbf{Purpose:} & Prints the stats of all players and creatures in the game, see sample run below for exact format. Each character's stats are separated by a dashed line, ``\verb|----------------------------------------------|". \\ \hline

\textbf{Parameters:} & 
This member function takes no parameters. \\ \hline

\textbf{Return Value:} & 
\mintinline{c++}{void} : The function does not return any value. \\ \hline

\textbf{Example:} & 

\begin{example}

\begin{minted}[breaklines=true]{c++}

int main() {
    Character player1("Elise", 100.0, 50, 'A', false);
    Character player2("Pawin", 80.0, 30, 'A', false);
    Character creature1("Goblin", 60.0, 20, 'C', true);

    Character players[2] = {player1, player2};
    Character creatures[1] = {creature1};

    Game game(players, creatures, 2, 1);
    game.printAllStats();
}
\end{minted}
\end{example}

\textbf{Expected Output:} 

\vspace{5pt}
Elise's stats: \newline
Health: 100.00 \newline
Status: A \newline
Mana: 50 \newline
Is Creature: No \newline
\noindent \verb|----------------------------------------------| \newline
Pawin's stats: \newline
Health: 80.00 \newline
Status: A \newline
Mana: 30 \newline
Is Creature: No \newline
\noindent \verb|----------------------------------------------| \newline
Goblin's stats: \newline
Health: 60.00 \newline
Status: C \newline
Mana: 20 \newline
Is Creature: Yes \newline
\noindent \verb|----------------------------------------------| \newline

\vspace{5pt} \\ \hline
\end{longtable}


%%%%%%%%%%%%%%%%%%Q6%%%%%%%%%%%%%%%%%%%%%%%%%%

\subsection{Loading Characters}

To begin the game, load characters from text files to populate the player and creature lists.

Create a players.txt file with the following content:

\begin{verbatim}
name|health|mana|status
Brian|100|50|A
Jenna|80|100|F
\end{verbatim}


Create a creatures.txt file with the following content:

\begin{verbatim}
name|health|mana|status
Goblin|100|50|A
Frog|80|100|A
\end{verbatim}

Note: All files used in this problem are in a zipped folder titled \texttt{'homework\_8\_input\_files.zip'} located in Canvas under the week 11 module. \newline

Write a function \mintinline{c++}{loadCharacters()} that reads data from a text file and fills an array of Character objects. The text file contains information about different characters, with each character's information separated by the character | and each character listed on a new line. The function specification is provided below. \newline



\textbf{For Question 6, develop and validate your solution in VS Code. Once you are happy with your solution, go to coderunner on Canvas and paste \mintinline{c++}{Character.h}, \mintinline{c++}{Character.cpp}, \mintinline{c++}{Game.h}, \mintinline{c++}{Game.cpp}, and \mintinline{c++}{loadCharacters} function into the answer box.} 

\renewcommand{\arraystretch}{1.5}
\begin{longtable}{|p{1.7in}|p{4.3in}|}
\hline
\textbf{Function:} \mintinline{c++}{loadCharacters(string,} \newline \mintinline{c++}{Character[], const int,} \newline \mintinline{c++}{bool)} & \mintinline{c++}{bool loadCharacters(string filename, Character characters[],} \newline \mintinline{c++}{const int CHARACTERS_SIZE, bool is_creature)} \\ \hline

\textbf{Purpose:} & Reads data from the specified text file and fills the \texttt{characters} array with \texttt{Character} objects. Sets \mintinline{c++}{_isCreature} based on the \texttt{is\_creature} parameter. \\ \hline

\textbf{Parameters:} & 
\textbf{string} \texttt{filename} - Name of the text file to be read. \newline
\textbf{Character} \texttt{characters[]} - Array of \texttt{Character} objects to be filled. \newline
\textbf{const int} \texttt{CHARACTERS\_SIZE} - The maximum size of the \texttt{characters} array. \newline
\textbf{bool} \texttt{is\_creature} - Specifies if each character read from the file is a creature. \\ \hline

\textbf{Return Value:} & \texttt{bool} : Returns \texttt{true} if characters were successfully read and stored in the array; \texttt{false} if the file cannot be opened. \\ \hline

\textbf{Error Handling:} & 
- Returns \texttt{false} if the file cannot be opened. \newline
- Ignores any empty lines in the file. \newline
- The first line of the file is ignored as it contains headers. \\ \hline
\end{longtable}

\newpage
\begin{longtable}{|p{1.7in}|p{4.3in}|}
\hline
\textbf{Example:} & 
Note: This is only an example usage of the function; you need to develop your own function to fulfill the requirement for this problem.

\textbf{Sample text file:}

\begin{verbatim}
name|health|mana|status
Brian|100|50|A
Jenna|80|100|F 
\end{verbatim}

\vspace{5pt}

\begin{example}
\begin{minted}[breaklines=true]{c++}

int main() {
    string filename = "characters.txt";
    const int CHARACTERS_SIZE = 5;
    Character characters[CHARACTERS_SIZE];
    bool is_creature = false;
    cout << "Function returned value: " << loadCharacters(filename, characters, CHARACTERS_SIZE, is_creature) << endl << endl;
    // Print the contents of the characters array
    for (int i = 0; i < CHARACTERS_SIZE; i++) {
        if (characters[i].getName() != "") {
            characters[i].printStats();
            cout << endl;
        }
    }
}

\end{minted}
\end{example}

\textbf{Expected Output:} 

\vspace{5pt}
Function returned value: 1 \newline

Brian's stats: \newline
Health: 100.00 \newline
Mana: 50 \newline
Status: A \newline
Is Creature: No \newline
\noindent \verb|----------------------------------------------| \newline
Jenna's stats: \newline
Health: 80.00 \newline
Mana: 100 \newline
Status: F \newline
Is Creature: No
\vspace{5pt} \\ \hline

\end{longtable}








&nbsp;&nbsp;&nbsp;&nbsp;
## PreQuiz

\begin{problem}
    True or False: The following is valid C++ and will not return an error.

#include <iostream>
using namespace std;
int main()
{
bool intArray[3] = {true,false,false,false,true};
}

\end{problem}

\begin{problem}
    True or False: The following is valid C++ and will not return an error.

#include <iostream>
using namespace std;
int main()
{
bool intArray[7] = {true,false,false,false,true};
}

\end{problem}

\begin{problem}
    The program prints the contents of a string array. Fill in the blank accordingly:

    \begin{minted}[breaklines=true]{c++}
#include <iostream>
using namespace std;

int main()
{
    // Create a character array of size 6 and initialize it with the characters 'h', 'e', 'l', 'l', 'o', '\0'
    _____________________________ // FILL IN THIS LINE

    cout << "The contents of the array are: ";
    for(int i = 0; i < 5; i++)
    {
        cout << message[i] << " ";
    }
    return 0; 
}


\end{problem}

\begin{problem}
The program below creates an array of strings and then prints the first letter of every string, the second letter of every string, so on and so forth. Fill in the blank accordingly.

\begin{minted}[breaklines=true]{c++}
#include<iostream>
using namespace std;

int main()
{
    string fruits[] = {"Apple", "Banana", "Cherry", "Date", "Fig", "Grape", "Mango"};

    int longest_string = fruits[0].length();
    for (int i = 1; i < 7; i++){
        if (______________){ //FILL IN THIS LINE
            longest_string = fruits[i].length();
        }
    }

    for (int i = 0; i < longest_string; i++){
        for (int j = 0; j < 7; j++){ 
            if (i < static_cast<int>(fruits[j].length())){
                cout << ____________ << endl; // FILL IN THIS LINE
            }
        }
    }

}


\end{problem}

&nbsp;&nbsp;&nbsp;&nbsp;
## Recitation

&nbsp;&nbsp;&nbsp;
### Spot The Error
\begin{multipart}
Given two positive integers `x` and `y`, this programs prints all the integer points (i, j) in the rectangle formed by (0, 0) and (x, y). Identify the error(s) in the code below, and write the correct line(s).
\end{multipart}


    #include <iostream>
    using namespace std;
    
    int main() 
    {
        int x = 3, y = 4;
    
        for(int i = 0; i >= x; j++)
        {
            for(int j = 0; j <= y; j++)
            {
                cout << "(" << i << ", " << j << ")  ";
            }
            cout << endl;
        }
    
    }


\begin{multipart}
The program prints the contents of an array and then calculates the sum of all the elements. Identify the error(s) in the code below, and write the correct line(s).
\end{multipart}


    #include <iostream>
    using namespace std;
    
    int main()
    {
        int numbers[5] = {10, 20, 30, 40, 50};
    
        cout << "The contents of the array are: ";
        for (int i = 0; i <= 5; i++)
        {
            cout << numbers << " ";
        }
        cout << endl;
    
        for (int i = 0; i <= 5; i++)
        {
            int sum = 0;
            sum += numbers;
        }
    
        cout << "Sum = " << sum << endl;
        return 0;
    }


\newpage

\begin{multipart}
The program finds and prints all prime factors of a given number `num`. Identify the error(s) in the code below, and write the correct line(s).
\end{multipart}


    #include <iostream>
    #include <cmath>
    
    using namespace std; 
      
    void primeFactors(int num)
    { 
        int n;
        while (n % 2 == 0) 
        { 
            cout << 2; 
            n = n / 2; 
        } 
    
        for (int i = 3; i <= sqrt(n); i++)
        { 
            while (n % i == 0) 
            { 
                cout << i << " "; 
                n = n / i; 
            } 
        } 
      
        if (n > 2) 
        {
            cout << n; 
        }
        cout<<endl;
    } 
    
    int main() 
    { 
        int num = 315; 
        primeFactors(num); 
        return 0; 
    }


\begin{multipart}
 The program prints the product of the length of contents of a string array. Identify the error(s) in the code below, and write the correct line(s).
\end{multipart}


    #include <iostream>
    #include <string>
    using namespace std;
    
    int main()
    {
        string languages[6] = {"C++","Python","Java","Matlab","Julia"};
        int product = 0;
        int total = languages.length();
    
        for(int i = 0; i <= total; i++)
        {
            product *= languages[i].length;
        }
    
        cout << "Product of lengths = " << product << endl;
        return 0;
    }


&nbsp;&nbsp;&nbsp;
### Combinations

Create two integer arrays `set1` and `set2` in the `main()`. The length of `set1` and `set2` should be 5 and 2, respectively. Prompt the user to enter 5 integers that go into `set1`. Do the same with 2 integers for `set2`. Then use the arrays to print all the possible pairs of the elements in `set1` with the elements in `set2`.

**Example output** (red is user input):

<div markdown="ol" style="margin-bottom: 10px; margin-top: 10px; overflow: hidden; color: #ffffff; background-color:rgb(6, 6, 6); border-color: #bce8f1; padding: 15px; border: 1px solid transparent; border-radius: 4px;">
Please enter 5 integers for the first set: \\
<span style="color:red">1 2 3 4 5</span>\\
Please enter 2 integers for the second set: \\
<span style="color:red">10 20</span>\\
1-10 1-20\\
2-10 2-20\\
3-10 3-20\\
4-10 4-20\\
5-10 5-20
</div>


\begin{multipart}
    Write out the steps you would use to solve this problem by hand as pseudocode. 
\end{multipart}

\newpage

\begin{multipart}
Pick possible inputs for your program. Choose as many inputs as you think you need to thoroughly test your program. Follow the steps you wrote for these values to find your result, and verify it.
\end{multipart}

\vspace{14cm}

\begin{multipart}
Implement your solution in C++ using VS Code. Revise your solution, save, compile and run it again. Are you getting the expected result and output? Keep revising until you do. Make you sure you test for the values used in your sample runs, and for the boundary conditions.
\end{multipart}

%no homework this week
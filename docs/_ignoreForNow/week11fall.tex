








&nbsp;&nbsp;&nbsp;&nbsp;
## PreQuiz
\begin{problem}
True or False: In C++, you must include a semicolon at the end of both struct and class definitions.
\end{problem}

\begin{problem}
True or False: In a class, public members are only accessible by member functions within the same class.
\end{problem}

\begin{problem}
True or False: Structs in C++ can only contain public data members, while classes can have both public and private data members.
\end{problem}

\begin{problem}
Fill in the blanks in the code below:

#include <iostream>
#include <string>
using namespace std;

struct ____ {  // Struct name
    string ____;  // Member for the name of the location
    int ____;     // Member for the population
};

int main() {
    ____ myLocation{"____", ____};  // Initialize with name and population

    cout << "Location: " << myLocation.____ << endl;
    cout << "Population: " << myLocation.____ << endl;

    return 0;
}

    
\end{problem}

\begin{problem}
Fill in the blanks in the code below:

#include <iostream>
using namespace std;

class ____ {  // Class name
public:
    ____ (int a) {  // Constructor to initialize age
        ____ = a;
    }
    
    void ____ (int a) {  // Setter method for age
        age = ____;
    }
    
    int ____() const {  // Getter method for age
        return ____;
    }

private:
    int ____;  // Private data member to store age
};

int main() {
    ____ person(25);  // Create a Person object with age 25
    person.____(30);  // Set the person's age to 30
    cout << "The person's age is: " << person.____() << endl;  // Get and display the age
    return 0;
}

\end{problem}

&nbsp;&nbsp;&nbsp;&nbsp;
## Recitation
&nbsp;&nbsp;&nbsp;
### Spot the Error
\begin{multipart}
The following code defines a class with a getter method to return a private member. Identify and fix the error.
\end{multipart}


#include <iostream>
using namespace std;

class MyClass {
public:
    MyClass(int val) { value = val; }

    int getValue() { return value; }

private:
    int value;
};

int main() {
    const MyClass obj(42);
    cout << obj.getValue() << endl;
    return 0;
}


\begin{multipart}
The following code is supposed to define a class with a constructor to initialize a private data member. Identify and fix the error.
\end{multipart}


#include <iostream>
using namespace std;

class MyClass {
public:
    MyClass(int val) { valueX = val; }

    int getValue() const { return value; }

private:
    int value;
};

int main() {
    MyClass obj(10);
    cout << obj.getValue() << endl;
    return 0;
}


\begin{multipart}
The following code is supposed to define a class with a private data member and a setter method. Identify and fix the error.
\end{multipart}


#include <iostream>
using namespace std;

class MyClass {
    int value;

    void setValue(int newValue) {
        value = newValue;
    }
};

int main() {
    MyClass obj;
    obj.setValue(20);
    return 0;
}



&nbsp;&nbsp;&nbsp;
### Coffee Shop Order System
This exercise focuses on using structs and basic arrays in C++ to create a simple coffee shop ordering system. You will define structs to represent drinks and orders, as well as functions to manage and calculate the order's total.

To begin, define a struct called <tt>Drink</tt>. The <tt>Drink</tt> struct should have three fields: a <tt>string</tt> <tt>name</tt> for the drink’s name (e.g., "Latte" or "Espresso"), a <tt>string</tt> <tt>size</tt> for the drink’s size ("small," "medium," or "large"), and a <tt>double</tt> <tt>price</tt> for the drink’s cost. Next, create a function called <tt>displayDrink</tt> that takes a <tt>Drink</tt> object as an argument and outputs its details in a readable format such as:

\begin{quote}
<tt>[Size] [Name]: \$[Price]</tt>
\end{quote}

For example, a medium latte priced at \$3.50 should display as <tt>Medium Latte: \$3.50</tt>.

Now, create a second struct called <tt>Order</tt> to represent a customer’s order. The <tt>Order</tt> struct should contain an array of <tt>Drink</tt> objects and an integer <tt>numDrinks</tt> that tracks the number of drinks in the order. To keep things simple, assume a maximum of 10 drinks per order. Define <tt>numDrinks</tt> with an initial value of 0.

Next, write a function called <tt>addDrink</tt> that takes an <tt>Order</tt> and a <tt>Drink</tt> as arguments, adds the drink to the order’s <tt>drinks</tt> array, and increments <tt>numDrinks</tt>. If the order already has 10 drinks, output an error message and prevent additional drinks from being added.

Now create a function called <tt>calculateTotal</tt> that takes an <tt>Order</tt> as a parameter and returns the total price of all drinks in the order by summing the <tt>price</tt> of each <tt>Drink</tt> in the <tt>drinks</tt> array. Finally, implement a function called <tt>displayOrder</tt> that prints out each drink in the order using <tt>displayDrink</tt> and then displays the total order cost.

In the <tt>main</tt> function, test your structs and functions by creating a few <tt>Drink</tt> objects, an <tt>Order</tt> object, and adding the drinks to the order. Finally, call <tt>displayOrder</tt> to print the details of the order and verify that the total is calculated correctly.

\begin{multipart}
    Write out the steps you would use to solve this problem by hand as pseudocode.
\end{multipart}

\vspace{8cm}

\begin{multipart}
    Write three possible lines you can include in your file to later test your program. Try to pick values that will test different aspects of your program. Follow the steps you wrote for these values to find your result, and verify it.
\end{multipart}

\vspace{5cm}

\begin{multipart}
    Implement your solution in C++ using VS Code. Save, compile, and run it. Test the program with different input files, including boundary conditions.
\end{multipart}

\vspace{5cm}














\vspace{10pt}

The Library class comprises the following attributes:

\vspace{10pt}

 & 


\begin{example}



\end{example}

 & 

\begin{example}


\end{example}

& 

\begin{example}


\end{example}



& 

\begin{example}


\end{example}


 
 & 

\begin{example}


**Example:** & 

\begin{example}


\newpage

\vspace{10pt}

\end{longtable}






































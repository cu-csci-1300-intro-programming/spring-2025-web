
&nbsp;&nbsp;&nbsp;&nbsp;
## PreQuiz

\begin{problem}
    Select True or False:
    \begin{enumerate}[label=\Alph*)]
        1. \textbf{T/F:} When writing switch statements in C++ you must include a default case, otherwise your switch statement is not valid.
        1. \textbf{T/F:} Switch statements can be built on integers, floats, and characters.
        1. \textbf{T/F:} There is no difference between an if/else-if/else-if chain and an if/if/if chain in C++.
    \end{enumerate}
\end{problem}

\begin{problem}
    Given two conditions, `cond1} and \mintinline{c++}{cond2}, the expression if \mintinline{c++}{(cond1 && cond2)} will evaluate to true only if \textunderscore \textunderscore \textunderscore \textunderscore \textunderscore \mintinline{c++}{cond1} and \mintinline{c++}{cond2} are true. On the other hand, the expression if \mintinline{c++}{(cond1 || cond2)} will evaluate to true if \textunderscore \textunderscore \textunderscore \textunderscore \textunderscore \mintinline{c++}{cond1} or \mintinline{c++}{cond2` is true.
\end{problem}

\begin{problem}
    How can you write a switch statement where multiple cases will execute the same block of code? Provide an example.
\end{problem}

\vspace{5cm}

\begin{problem}
    Fill in the blank(s) for the code below:
    
    int choice;
    cout << "Help our adventurer discover what to do next! Enter 1, 2, or 3." << endl;
    ______ >> choice;

    switch(____________){
        case ________:
            cout << "You found a hidden treasure!" << endl;
            break;
        case 2:
            cout << "You discovered a secret passage!" << endl;
            ___________
        case 3:
            cout << "You found a mystical artifact!" << endl;
            break;
        ________:
            cout << "That's not a valid case. Please try again." << endl;
    }
    
\end{problem}



&nbsp;&nbsp;&nbsp;&nbsp;
## Recitation

&nbsp;&nbsp;&nbsp;
### Spot The Error

\begin{multipart}
Below is code that asks the user for the day of the week as a number (Monday is 1, Sunday is 7) and then prints a corresponding statement. Identify the error(s):

    int day;
    cout << "What number day of the week is it?" << endl;
    cin >> day;
    switch (day) {
      case '6':
        cout << "Today is Saturday";
        break;
      case 7:
        cout << "Today is Sunday";
        
      default:
        cout << "Looking forward to the Weekend";
    }


\end{multipart}

\begin{multipart}
    Below is code with the same goal as the previous question, but different error(s). Identify the error(s):
    
    int day = 4;
    switch (day) 
      case 6:
        cout << "Today is Saturday";
        break;
      case 7:
        cout << "Today is Sunday";
        break;
      default
        cout << "Looking forward to the Weekend";
        
    
\end{multipart}

\begin{multipart}
    The code below is meant to determine if an angle is acute, obtuse, or right. Spot the error(s):
    
    #include <iostream>
    using namespace std;
    
    int main()
    {
        int angle =40;
        if (x<90) { 
            cout<<"It is an acute angle." ;
        }
        else if(x=90) {
            cout<<"It is a right angle.";
        }
        els{
            cout<<"It is an obtuse angle.";
        }
    }
    
\end{multipart}

\begin{multipart}
    The code below implements an exclusive OR logical operation, which means that only one of the conditions may be true. Spot the error(s):
    
    // This program implements XOR
    #include iostream
    using namespace std;
    
    //Set the variable value to 1 when x or y is 1
    int main(){
        int x = 1,y=0,value;
        
        if (x == 1){ 
            if(y==0)
            value = 1; 
    
            else
            y == 0; 
         
        if(x==0){ 
            if(y==0)
            value = 0; 
    
            else
            value = 1;
        }
        
        cout < value < endl;
        return 0;
    }
    
\end{multipart}

&nbsp;&nbsp;&nbsp;
### Final Velocity of a Rocket

Write a C++ program that will calculate the final velocity of a rocket after 20 seconds. The program will ask the user for the initial velocity (m/s) and the fuel type (A, B, C). The rate of acceleration will depend on the type of fuel and the initial velocity.


    1. If initial velocity is less than 10, then the acceleration rate for each fuel type is as follows
    
        1. Fuel type A $\rightarrow$ 5 (m/s) per second
        1. Fuel type B $\rightarrow$ 10 (m/s) per second
        1. Fuel type C $\rightarrow$ 20 (m/s) per second
    
   1. If initial velocity is greater than or equal to 10 and less than or equal to 40, then the acceleration rate
   for each fuel type is as follows
   
        1. Fuel type A $\rightarrow$ 6 (m/s) per second
        1. Fuel type B $\rightarrow$ 12 (m/s) per second
        1. Fuel type C $\rightarrow$ 24 (m/s) per second
   
    1. If initial velocity is greater than 40, then the acceleration rate for each fuel type is as follows
    
        1. Fuel type A $\rightarrow$ 3 (m/s) per second
        1. Fuel type B $\rightarrow$ 6 (m/s) per second
        1. Fuel type C $\rightarrow$ 9 (m/s) per second 
    


Below are some sample runs. User input is shown in bold. 

<div markdown="ol" style="margin-bottom: 10px; margin-top: 10px; overflow: hidden; color: #ffffff; background-color:rgb(6, 6, 6); border-color: #bce8f1; padding: 15px; border: 1px solid transparent; border-radius: 4px;">
    Enter the initial velocity:
    
    \textbf{70}
    
    Enter the fuel type:
    
    \textbf{C}
    
    The final speed is 250 m/s.
</div>

<div markdown="ol" style="margin-bottom: 10px; margin-top: 10px; overflow: hidden; color: #ffffff; background-color:rgb(6, 6, 6); border-color: #bce8f1; padding: 15px; border: 1px solid transparent; border-radius: 4px;">
    Enter the initial velocity:
    
    \textbf{5}
    
    Enter the fuel type:
    
    \textbf{A}
    
    The final velocity is 105 m/s.
</div>

\newpage

\begin{multipart}
    Write out the steps you would use to solve this problem by hand as pseudocode. 
\end{multipart}

\vspace{10cm}

\begin{multipart}
    Pick possible inputs for your program. Follow the steps you wrote for these values to find your result, and verify it.
\end{multipart}

\vspace{3.5cm}

\begin{multipart}
     Identify two possible values that are ``boundaries" in this problem that you will have to test. What should happen for these values?
\end{multipart}

\vspace{3.5cm}

\begin{multipart}
    Translate your pseudocode into a c++ program to solve the above code.
\end{multipart}

